\documentclass[12pt]{article}
\usepackage{hyperref}
\usepackage{amsmath}
\usepackage{amssymb}
\begin{document}
\[ \text{Prakash Pant} \]
\[ \text{ University of Vermont} \]
\[ \text{ Bardiya, Nepal} \]
\[ \text{ Email =\{ ``Prakash.Pant@uvm.edu'',``prakashpant.np@gmail.com'' \} }  \]

\[ \]

Alert:  Some Unsolved Problems I discovered througout the video making journey \\
1. 
$\sum_{n=1}^{\infty} (-1)^n \frac{1}{p_n}$ where $p_n$ is nth prime    \\ 2. Finding trigamma(1/n) summation formula

\[ \]


\textbf{Note: All the problems below have a solution video on \href{https://www.youtube.com/@mathalysisworld}{``Mathalysis World"} youtube channel. You can just click on the `solution' that appears at the right of every problem to access solution. Problems that appear early have comparatively lower quality solution video.}

\[ \footnote{A surprisingly easy Geometric Integral}  \int_0^\infty 0.5^{\lfloor x \rfloor}  dx  = 2   \href{https://youtu.be/BXQjZkRU1bc?si=4nXwmIqkPmGYgrVY}{\,\,=>solution} \]

\[ \footnote{King comes to your help} \int_0^\frac{\pi}{4}	\frac{(\sin{x}+\cos{x})}{(9+16\sin{2x})} dx 	= \frac{\ln{3}}{20} \href{https://youtu.be/phRpCE80-H8?si=MhZupjfRLEAfoZAx}{\,\,=>solution}
\] 

\[\footnote{Is MIT integration Bee this easy?}  \int	\frac{(x+1)}{x(x+\ln{x})} dx  = \ln{(x+\ln{x})}+c			\href{https://youtu.be/yDneoqTMABM?si=Pg4eIet-QmkqFSZj}{\,\,=>solution}					
\] 

\[\footnote{The Mysterious Integral}  \int_0^1	\frac{\ln(x+1)}{x^2+1} dx 	= \frac{\pi\ln{(2)}}{8}	\href{https://youtu.be/Oj6Uyg74Nxw?si=nj_UMz-jQcI9wkFj}{\,\,=>solution} 							
\] 

\[\footnote{The Quarrelsome Integral} \int_0^1	(\sqrt[2022]{1-x^{2020}} - \sqrt[2020]{1-x^{2022}} )  dx  =0 								 
\href{https://youtu.be/MYRzIZoUP8o?si=YWMqv4JIZ5kZTYxN}{\,\,=>solution}   \] 

\[\footnote{Sophomore's Dream-i} \int_0^1 x^x dx = \sum_{n=1}^{\infty} \frac{(-1)^{n-1}}{n^n} 
\href{https://youtu.be/Qk1CrunOH6A?si=SSp4823NQYmGg7X1}{\,\,=>solution}   \]

\[\footnote{Sophomore's Dream-ii} \int_0^1 x^{-x} dx = \sum_{n=1}^{\infty} \frac{1}{n^n} 
\href{https://youtu.be/Qk1CrunOH6A?si=SSp4823NQYmGg7X1}{\,\,=>solution}  \]

\[\footnote{The trigonometric towers integral }   \int_{\pi/6}^{\pi/3} ({\sin{x}^{\cos{x}^{\sin{x}}}} - {\cos{x}^{\sin{x}^{\cos{x}}}}) dx  = 0  \href{https://youtu.be/j_ZJ-k4ew70?si=pD-khsNvuCNT6bgB}{\,\,=>solution}  \]

\[\footnote{The Trigonometric BUS Integral}  \int_{-\pi}^{\pi} (\sin{x} + 2 \sin{(2x)} + 3 \sin{(3x)} + 4 \sin{(4x)} + 5 \sin{(5x)} ) ^2 dx   = 55\pi \href{https://youtu.be/QOuvlXys16s?si=O6EeaHoO9gufNuOo}{\,\,=>solution}  \]

\[\footnote{Bro, Are you joking?}  \int_{0}^{\infty} \frac{1}{1+x+x^2+x^3+x^4+x^5} dx = \frac{\pi}{3\sqrt{3}} \href{https://youtu.be/fZov2OJl03Q?si=ULeqGBbU8lTifa8h}{\,\,=>solution}  \]

\[ \footnote{How easy is MIT Integration Bee?} \int tanh^2(x) dx = x-\tanh{x}  \href{https://youtu.be/TZjeZUDTwU0?si=m5_wBG3TkumZFYSv}{\,\,=>solution}  \]

\[ \footnote{Integrating using series in MIT Integration Bee} \int_0^1 \frac{\ln{(1+x)}}{x} dx =\frac{\pi^2}{12} \href{https://youtu.be/dbxeH9CWUdY?si=AkAtHU4RbzLiY5me}{\,\,=>solution}   \]

\[ \footnote{This is the most easiest difficult question in MIT Integration Bee} \int_0^1 \prod_{k=0}^\infty \left(\frac{1}{1+x^{2^k}}\right) dx   = \frac{1}{2} \href{https://youtu.be/Q6eLckbAkIc?si=mcF_DL4qtBiO-XVu}{\,\,=>solution}   \]

\[ \footnote{Chinese University Wifi Password} \int_{-2}^2 (x^3\cos{(\frac{x}{2})} + \frac{1}{2})\sqrt{4-x^2} dx   = \pi  \href{https://youtu.be/Kx6ntqc-3Ms?si=4fE58BKu-BP1rSAp}{\,\,=>solution}  \]

\[ \footnote{Beta Gamma Function in MIT Integration Bee} \int_0^{\pi/2} \frac{\sqrt[3]{\tan{(x)}}}{(\sin{(x)}+\cos{(x)})^2} dx =\frac{2\sqrt{3}\pi}{9}    \href{https://youtu.be/NcSVlFkmk9Y?si=iyaRQ5KFEY_iCGzq}{\,\,=>solution}  \]

\[ \footnote{Stepping on some hard integrals} \int_0^{1} \frac{x \ln{(x)}}{x^4+x^2+1} dx = \frac{1}{36} \left( \psi_1{(\frac{2}{3})} - \psi_1{(\frac{1}{3})} \right)   \href{https://youtu.be/oPBDgloQqIw?si=nQNZw3YS3i0cikBO}{\,\,=>solution}   \]


 \[ \footnote{ proof of why $\int_0^{\pi/2} \ln{(\sin{(y)})} dy = -\ln{(2)} \frac{\pi}{2} $ }  \int_0^{\pi/2}\ln{(\sin{(y)})}dy = -\ln{(2)} \frac{\pi}{2} \href{https://youtu.be/8JsRstQQ9XE?si=FYl7gDU_DN6ZZNVa}{\,\,=>solution}   \]

\[ \footnote{ proof of Dirichlet Integral}  \int_0^{\infty} \frac{\sin{(x)}}{x} dx  = \frac{\pi}{2} \href{https://youtu.be/w40ksfg-q0k?si=2P1emx2OLUxAgIRl}{\,\,=>solution}   \]

\[ \footnote{Two wonderful Series from Dr. Peyam-i} \ln{(2\sin{(x)})} = \sum_{n=1}^{\infty} - \frac{\cos{(2nx)}}{n} \href{https://youtu.be/OhjEuSlL2xM?si=yHkPODrOtvp_1Bm-}{\,\,=>solution}   \]
\[ \footnote{Two wonderful Series from Dr. Peyam-ii}
\ln{(2\cos{(x)})} = \sum_{n=1}^{\infty} (-1)^{n+1} \frac{\cos{(2nx)}}{n}  \href{https://youtu.be/OhjEuSlL2xM?si=yHkPODrOtvp_1Bm-}{\,\,=>solution}   \]

\[ \footnote{Best application of Beta Gamma Function } \int_0^{\frac{\pi}{2}} \sqrt{\tan{(x)}} dx \href{https://youtu.be/0QnZdO_NI3w?si=CuM191Mhh3utn00Z}{\,\,=>solution}   \]

\[ \footnote{Proof of Dirichlet Integral} \int_0^{\infty}  \frac{\sin{(x)}}{x} dx = \frac{\pi}{2}\href{https://youtu.be/w40ksfg-q0k?si=IlERIqHIwE3sJrJi}{\,\,=>solution}   \]
\[ \footnote{Proof of Generalized Dirichlet Integral} \int_{0}^{\infty} \frac{\sin{(x^n)}}{x^n} dx \href{https://youtu.be/EnK5X2zXAh0?si=dTmrqUZJxtJf5CB4}{\,\,=>solution}   \]
\[ \footnote{ Easiest way to prove Fresnel Integrals using Beta Gamma Functions} \int_0^{\infty} \sin{(x^2)} dx = \frac{\sqrt{\pi}}{2\sqrt{2}} \href{https://youtu.be/LLyCzkri8DA?si=w5ShWB6GzCRpt0pE}{\,\,=>solution}   \]
\[ \footnote{ Easiest way to prove Fresnel Integrals using Beta Gamma Functions} \int_0^{\infty} \cos{(x^2)} dx = \frac{\sqrt{\pi}}{2\sqrt{2}} \href{https://youtu.be/LLyCzkri8DA?si=w5ShWB6GzCRpt0pE}{\,\,=>solution}   \]
\[ \footnote{Generalized Fresnel Integral-i}  \int_0^{\infty} \sin{(x^n)} dx \href{https://youtu.be/SCeCd_ffPdk?si=aXK0-MZ9mtSvtsM3}{\,\,=>solution}   \]
\[ \footnote{Generalized Fresnel Integral-i}  \int_0^{\infty} \cos{(x^n)} dx \href{https://youtu.be/SCeCd_ffPdk?si=aXK0-MZ9mtSvtsM3}{\,\,=>solution}   \]
\[ \footnote{Solving Fresnel Integral from Laplace Transform} \int_0^{\infty} \sin{(x^2)} dx = \frac{\sqrt{\pi}}{2\sqrt{2}} \href{https://youtu.be/r7gn6qvsQ7k?si=1mdHc0BmQwlgE4qB}{\,\,=>solution}   \]

\[ \footnote{ Solving 20 integrals by Feynman Technique} \href{write your reference here}{\,\,=>solution}   \]

\[ \footnote{Another easy integral from MIT Integration Bee} \int_0^1 \frac{1+x^2}{1+x^4} dx  \href{https://youtu.be/5rXZp_p-7A0?si=a0B6WlE98buU3Q1d}{\,\,=>solution}   \]

\[ \footnote{The Catalan Integral} \int_0^1 \frac{\arctan{x}}{x} dx  \href{https://youtu.be/s8k_vT4Whm8?si=AzWPnSgTvpHGCTwk}{\,\,=>solution}   \]

\[ \footnote{Not the Dirichlet Integral} \int_0^{\frac{\pi}{2}} \frac{x}{\sin{(x)}} dx  \href{https://youtu.be/-hgapjphTLM?si=xILP-5hwMdw0F8_M}{\,\,=>solution}   \]

\[ \footnote{Shorts: What is log{(-2023)} ? } \log{(-2023)}  \href{write your reference here}{\,\,=>solution}  \] 

\[ \footnote{Let's go to Complex World } \int_1^{2023} \log{(-x)} dx \href{https://youtu.be/bPDoqb4BZZE?si=deYQAD7QiblS45pS}{\,\,=>solution}   \]

\[ \footnote{ The tower of x integral } \int_0^1 x^{x^{x^{x^{x^{x^{x^{x^{.^{.^{.^{.^{.^{.^{.^{.}}}}}}}}}}}}}}} dx   = Diverges \href{https://youtu.be/Qsrf7rkHK1c?si=EYU78HOb0OVWBgNv}{\,\,=>solution}   \]

\[ \footnote{ My take on your "A satisfying gamma function integral @maths505" } \int_{-\infty}^{\infty}\Gamma{(1+i x)}\Gamma{(1-ix)}dx=\frac{\pi}{2}  \href{https://youtu.be/fdUBfHPfhCk?si=1RY8Jz3dWcE7TTXI}{\,\,=>solution}   \]

\[   \footnote{ Integrating Lambert W function}  \int_0^e  W(z) dz  \href{https://youtu.be/d27ej5SU1gM?si=7PRUDrkf2jWeYLJi}{\,\,=>solution}   \]
\begin{center} where W(z) is Lambert W function  \end{center}

\[ \footnote{Deriving the series of Lambert W function} W(z)= \sum_{n=1}^{\infty} \frac{(-n)^{(n-1)} x^n}{n!}   \href{write your reference here}{\,\,=>solution}   \]

\[  \footnote{ The Tower of 1/e} {\frac{1}{e}^{\frac{1}{e}^{\frac{1}{e}^{\frac{1}{e}^{\frac{1}{e}^{\frac{1}{e}^{\frac{1}{e}^{.^{.^{.^{.}}}}}}}}}}}  = \Omega \href{https://youtu.be/skNiMnmqGyg?si=hLYI4T0DtJSfuL3U}{\,\,=>solution}    \]

\[  \footnote{Freaking Irrational Integral} \int_0^{\infty} \frac{x^e}{1+x^{2(e+1)}} dx  = \frac{\pi}{2(e+1)} \href{https://youtu.be/h-nTpipYkU0?si=GWwUOYfFpiZUWs5Y}{\,\,=>solution}   \]

\[ \footnote{Integration using Fourier Series} \int_0^{\frac{\pi}{4}} \log{(2\cos{(x)})} dx  \href{https://youtu.be/gpMaOXIcE_s?si=gY8jxdlcRJ6lJVeO}{\,\,=>solution}   \]

\[ \footnote{Most satisfying Double Integral } \int_0^1 \int_0^1 \frac{dx dy}{1+x^2y^2} \href{https://youtu.be/RSAZngTcIoQ?si=fD-uOPV2yXjotzqN}{\,\,=>solution}    \]

\[ \footnote{ I can solve the impossible Integral -i} \int_0^3 \int_{x^2}^9 x^3 e^{y^3} dy dx   \href{https://youtu.be/M9Ik9BM207I?si=sUFM-rGm0x31hqqc}{\,\,=>solution}    \]

\[ \footnote{ I can solve the impossible Integral - ii} \int_0^8 \int_{\sqrt[3]{y}}^2 \sqrt{x^4+1} dx dy  \href{https://youtu.be/yxlrFlZ9J50?si=OCxAnQMSausrwXn0}{\,\,=>solution}   \]

\[ \footnote{ Is that even possible?} y^{\frac{dy}{dx}} = e^y   \href{write your reference here}{\,\,=>solution}   \]

\[ \footnote{ MIT Integration Bee Qualifier Exam P10} \int ((1-x)^3+(x-x^2)^3+(x^2-1)^3-3(1-x)(x-x^2)(x^2-1)) dx  = 0    \href{write your reference here}{\,\,=>solution}   \]

\[ \footnote{Dear @Maths505, here's my approach} \int_0^{\frac{\pi}{2}} \frac{\ln{(\sec{(x)})}}{\tan{(x)}} dx = \frac{\zeta{(2)}}{4}  \href{https://youtu.be/d9YUD7Z6PP4?si=9M7VzsV1lmIGqFqz}{\,\,=>solution}   \]

\[  \footnote{ Proving using Beta Gamma Function \\ You can not have a more difficult proof than this}  \sum_{n=0}^{\infty} \frac{1}{n!} = e     \href{https://youtu.be/1XuxtJsd4Lg?si=hcWPqC7pyTcMNRB2}{\,\,=>solution}   \]

\[  \footnote{Using symmetricity in Integrals} \int_0^1 \int_0^1 \frac{xy\sqrt{x}}{x\sqrt{y}+y\sqrt{x}} dx dy   \href{https://youtu.be/yD2IxWw7iSc?si=VltNzxyx9Z_g9Baw}{\,\,=>solution}   \]

\[ \footnote{A standard technique for such problems: use generating function for harmonic number} \sum_{n=1}^{\infty} \frac{H_n}{2^n} = ln{(4)}    \href{https://youtu.be/GsYc6RyozE4?si=vvfjZd4EiTiaNYqb}{\,\,=>solution}   \]

\[ \footnote{This is the best use of Feynman's Method} \int_0^{\infty} \frac{e^{-t}-e^{-tx}}{t} dt = \ln{(x)}  \href{https://youtu.be/HdknWymWaqU?si=3VBwBX_VacoHNuhf}{\,\,=>solution}   \]

\[ \text{ Words of an adolescent} \href{https://youtu.be/9z6-O_OtfDs?si=W4haN6krrNjYT-9r}{\,\,=>solution} \]
\[ \footnote{Origin of Gamma Function} \int_0^{\infty} e^{-t} t^{x-1} dx = \Gamma{(x)} \href{write your reference here}{\,\,=> solution}  \]

\[ \footnote{Origin of Beta Function} \int_0^{\infty} t^{m-1} (1-t)^{n-1} dt = \beta{(m,n)} \href{write your reference here}{\,\,=> solution}   \]

\[ \footnote{Solving the easiest integral using hardest technique i.e. Ramanujan's Master Theorem } \href{write your reference here}{\,\,=>solution}   \]

\[  \footnote{ You cannot get a more easier integral than this in MIT Integration Bee} \int_{-\frac{\pi}{2}}^{\frac{\pi}{2}} \frac{1}{1+e^{x\cos{(x)}}} dx   \href{https://youtu.be/DuIY_eahtdw?si=F3SvFZdjPbaomf64}{\,\,=>solution}    \]

\[ \footnote{ The Legend of JEE Mains solved by only  5 percent students } \int_{-\frac{\pi}{4}}^{\frac{\pi}{4}} \frac{1}{(1+e^{x\cos{x}}) (sin^4x+cos^4x)}   \href{https://youtu.be/tHzIJAzB3J8?si=3DoF9JjLyxjZBNI0}{\,\,=>solution}   dx\]
   \[ \footnote{ This is the best use of Lambert W function } \sqrt{2}^{\sqrt{2}^{\sqrt{2}^{\sqrt{2}^{\sqrt{2}^{\sqrt{2}^{\sqrt{2}^{\sqrt{2}^{\sqrt{2}^{\sqrt{2}^{\sqrt{2}^{.}^{.}^{.}^{.}  }}}}}}}}}}  \href{write your reference here}{\,\,=>solution}    \] 

\[ \footnote{Deriving the formula for volume of n dimensional sphere  }  V_n = \frac{\pi^{\frac{n}{2}}}{(\frac{n}{2})!} r^n  \href{write your reference here}{\,\,=>solution}   \]

\[ \footnote{ sum of the volumes of all n-dimensional spheres}   \sum_{n=2k}^{\infty} V_n = e^{\pi} \href{https://youtu.be/iO7paZQZuHY?si=msBOq9ttAJODHM1O}{\,\,=>solution}   \]  where $V_n$ is volume of n dimensional Sphere 

\[ \footnote{ sum of the volume of even dimensional spheres } \href{https://youtu.be/NEp43AnrVu0?si=XO_tq5OQTJDGunKb}{\,\,=>solution}    \]

\[ \footnote{ A Complex triggy boi} \int_0^{\frac{\pi}{2}} \tan{^ix} \, dx \href{https://youtu.be/ALsO97b50fo?si=hdcKNqf83vkz6Kc7}{\,\, => solution}   \]

\[ \footnote{ Using Laplace Transform to solve for an absolutely gorgeous result } \int_{-\infty}^{\infty} \frac{\cos{(x)}}{x^2+1} dx  \href{https://youtu.be/V6XhFcQZZPQ?si=y2wNdsMPp7EmQkDO}{\,\,=> solution}   \]

\[ \footnote{Short Animation Proof of this absolutely gorgeous result} \gamma= \sum_{m=2}^{\infty} (-1)^m \frac{\zeta(m)}{m} \href{https://youtu.be/_tfxR8zqAb0?si=b6GfXRX8hGphHhHn}{\,\,=>solution}  \]

\[ \footnote{ MIN Integration Bee 2010 Qualifier Problem 8} \int_1^{\infty} \frac{dx}{x\sqrt{x^4-1}}  \href{https://youtu.be/u6s9K5lmssM?si=x5hggLXoGm_n_3MD}{\,\,=>solution}   \]

\[ \footnote{ MIT Integration Bee 2010 Qualifier Problem 25} \int_1^2 (x-1)^{\frac{1}{2}} (2-x)^{\frac{1}{2}} dx  \href{https://youtu.be/v3AeIqXh7No?si=9uWBV7CUXrxZ56-z}{\,\,=>solution}  \]

\[ \footnote{ An awesome limit problem} \lim_{x\to\frac{\pi}{4}} (1+\sin(x)-\cos(x))^{\tan(2x)} = e^{\frac{-1}{\sqrt{2}}} \href{https://youtu.be/DZ_Id8-SbhQ?si=H_b3GCLzU4TaVEbG}{\,\,=>solution}  \]

\[ \footnote{ How high school student vs University student solve this limit?} \lim_{x \to 0} \frac{1-x \cot x}{x^2} \href{https://youtu.be/COP94ZehPQw?si=Vn-cowczXyFsO1pk}{\,\,=>solution}  \]

\[ \footnote{ A single liner solution using Maz Identity} \int_{-\infty}^{\infty} \frac{\sin(x)}{x} dx  \href{https://youtu.be/y6vqqgjhjkQ?si=VvRBAZfoHIpIJs9-}{\,\,=>solution}   \]

\[ \footnote{ How Undergrad. Vs Grad solve this integral?} \int_{-\infty}^{\infty} \frac{1-\cos x}{x^2} dx \href{https://youtu.be/xe2u1_g1vHg?si=igqkwMN04zKUcE1K}{\,\,=>solution}  \]

\[ \footnote{ Unseemingly hard Quadratic equation} \frac{1}{a}+\frac{1}{b}+\frac{1}{x}=\frac{1}{a+b+x}  \href{https://youtu.be/6uN_Iw6TRzE?si=NF6YZOh87MfY0yd8}{\,\,=>solution}   \]

\[ \footnote{ WTF are these things?)} \frac{d^{-1}}{dx^{-1}}(x)   \frac{d^{\frac{1}{2}}}{dx^{\frac{1}{2}}}(x)    \frac{d^{\frac{1}{2}}}{dx^{\frac{1}{2}}}(1) \href{https://youtu.be/Xr1Dv8_kg8A?si=qMXY6SZsLZCjRiov}{\,\,=>solution}   \] 

\[ \footnote{ How come we have cot here?} \sum_{n=1}^{\infty} \frac{1}{n^2-x^2} = \frac{\pi \cot(\pi x)}{-2x}+\frac{1}{2x^2}  \href{https://youtu.be/Dw7LwzKBCqg?si=71HQ5PRLGivewr-U}{\,\,=>solution}   \]

\[ \footnote{ Stanford Mathematics Tournament} \sum_{n=1}^{\infty} \frac{\zeta(2n)}{\pi^{2n}} \href{https://youtu.be/tiO_VwY6FrE?si=xzzbnWcDUBvi8ir9}{\,\,=>solution}   \]

\[ \footnote{ $x^{p-1}/e^x-1$ // Product of Eulers Gamma and Reimann zeta function interms of Bose integral} \int_0^{\infty} \frac{x^{p-1}}{e^x-1} dx = \Gamma(p) \zeta(p) \href{https://youtu.be/dk6KTuRrAQU?si=-bsuzuhqQ6BBwr08}{\,\,=>solution}   \]

\[ \footnote{ $x^{p-1}/e^x+1$ // Product of Eulers Gamma and Dirichlet eta function} \int_0^{\infty} \frac{x^{p-1}}{e^x+1} dx = \Gamma(p) \eta(p)  \href{https://youtu.be/dk6KTuRrAQU?si=-bsuzuhqQ6BBwr08}{\,\,=>solution}   \]

\[ \footnote{ Relation between Dirichlet Eta and Reimann Zeta Function} \eta(s)= (1-\frac{2}{2^s}\zeta(s)  \href{https://youtu.be/OyuavSaTP4A?si=tRRfX1QAGfr1v1DZ}{\,\,=>solution}   \]

\[ \footnote{MIT Integration Bee: This is the best application of MAZ Identity} \int_0^{\infty} \frac{\sin(x)}{x^{\frac{3}{2}}} dx  \href{https://youtu.be/gfmLXQVzbew?si=5vlyuZ3pIDCAvJTf}{\,\,=>solution}   \]

\[ \footnote{ Euler Representation of Gamma Function} \Gamma(x) = \int_0^{\infty} e^{-t} t^{x-1} dt   \href{https://youtu.be/S1YworuetsQ?si=ONX8GAi62LmMYQCR}{\,\,=>solution}   \]

\[ \footnote{ Gauss Representation of Gamma Function} \Gamma(s) = \lim_{n\to\infty} \frac{n^s}{s} \prod_{k=1}^{n} \frac{k}{s+k} \href{https://youtu.be/KA3o4-iayio?si=3NMx8QCkCRYQVvda}{\,\,=>solution}   \]

\[ \footnote{ Weierstrass Representation of Gamma Function} \frac{1}{\Gamma(x)}= x e^{\gamma x} \prod_{n=1}^{\infty} (1+\frac{x}{n}) e^{-\frac{x}{n}} \href{https://youtu.be/LfqnG71bVLw?si=Qi71ppaumBdRcB2l}{\,\,=>solution}   \]

\[ \footnote{Infinite Sum Representation for Digamma Function} \psi(x+1)= -\gamma + \sum_{k=1}^{\infty} \frac{1}{k}-\frac{1}{k+x} \href{https://youtu.be/QpXT6nnttjM?si=SoNhT1EEsvLYKRcA}{\,\,=>solution}   \]

\[ \footnote{ This is the most beautiful equation in mathematics, Deriving from Scratch} \psi(x+1)= -\gamma + H_n \href{https://youtu.be/P6ak6-CpXeY?si=Iyx_osJOqrf1GDdV}{\,\,=>solution}   \]


\[ \footnote{ Integral Representation for Digamma Function} \psi(x+1) = -\gamma + \int_0^{\infty} \frac{1-x^n}{1-x} dx \href{https://youtu.be/2FfOE_Pu3hg?si=NT6ioHsTMLz1fuO-}{\,\,=>solution}  \]

\[ \footnote{ Reflection formula for Digamma Function} \psi(1-n)-\psi(n)=\pi \cot(n\pi) \href{https://youtu.be/vMb-d-fjcSc?si=E6qFObf7L-seePSX}{\,\,=>solution}  \]
\[ \footnote{ Duplication formula for Digamma Function} 2 \psi(2m)= \psi(m)+\psi(m+\frac{1}{2})+2\ln(2) \href{https://youtu.be/vMb-d-fjcSc?si=E6qFObf7L-seePSX}{\,\,=>solution}   \]

\[ \footnote{ The classic Problem from MIT Integration BEE} \int_0^{\infty} (1-x\sin\left(\frac{1}{x}\right) ) dx  \href{https://youtu.be/NAGHF5Y0760?si=GY7_K_77Ka8YPOSz}{\,\,=>solution}   \]

\[ \footnote{ Harvard MIT Maths Tournament} \lim_{n\to\infty} \left( \frac{1}{\sqrt{n^2-0^2}}+\frac{1}{\sqrt{n^2-1^2}}+.....+\frac{1}{\sqrt{n^2-(n-1)^2}} \right)  \href{https://youtu.be/EhFEXeA9RIE?si=2C0KzZov-OS_0PP1}{\,\,=>solution}   \]

\[ \footnote{ A high school limit problem from IIT JEE} \lim_{x\to0} \left( \frac{\sqrt[x]{1+x}}{e} \right)^{\csc{x}}  \href{https://youtu.be/wGH_R0W4hlI?si=JzzKLlTU_wHIHInJ}{\,\,=>solution}   \]

\[ \footnote{ Limit involving Reimann Sum} \lim_{n\to\infty} \frac{1}{\sqrt{n}} \sum_{k=1}^{n} \frac{1}{\sqrt{n+k}}  \href{https://youtu.be/bXurKYvv9DY?si=rdPhQTzyl4akl7ac}{\,\,=>solution}   \]

\[ \footnote{ Two important infinite sums} \sum_{n=1}^{\infty} \frac{1}{n^2+x^2} \sum_{n=1}^{\infty} \frac{1}{n^2-x^2}  \href{https://youtu.be/YCaNUrtiqS0?si=VvzhGOBUKL8ZZGyG}{\,\,=>solution}   \]

\[ \footnote{Trivial Proof of Euler's Prime Product Formula // Relation between Reimann Zeta Function and Prime Numbers} \zeta(s) = \prod_{prime} \frac{1}{1-p^{-s}}  \href{https://youtu.be/Q340SuV68Ls?si=OwIn31SawSk6GdZl}{\,\,=>solution}   \]

\[ \footnote{ Two amazing theorems of MAZ -I} \int_0^{\infty} f(s)g(s) ds = \int_0^{\infty} \mathcal{L} \{ f \}(t)\,\, \mathcal{L}^{-1} \{g\} (t) dt  \href{https://youtu.be/-5zj_tiP8KM?si=ulWPg2vK5R8ahFCQ}{\,\,=>solution}  \]

\[ \footnote{ Two amazing theorems of MAZ -II} \sum_{n=1}^{\infty} f(n) = \int_0^{\infty} \frac{\mathcal{L}^{-1} \{ f\}(t)  }{e^t-1} dt  \href{https://youtu.be/-5zj_tiP8KM?si=ulWPg2vK5R8ahFCQ}{\,\,=>solution}  \]

\[ \footnote{ MAZ theorem helps me solve this infinite sum} \sum_{n=2}^{\infty} \frac{4n-3}{n(n^2-1)}  = \frac{9}{4}   \href{write your reference here}{\,\,=>solution}   \]

\[ \footnote{ Monstrous JEE Advanced Integral} \int_0^{\ln(2)} \frac{e^x - e^{2x} + e^{3x}-e^{4x}}{1+e^x + e^{2x} + e^{3x}} dx  \href{https://youtu.be/zW7x7ImDONY?si=hfmoj4D_B3FQmUA7}{\,\,=>solution}   \]

\[ \footnote{ MAZ theorem helps me solve this integral} \int_0^{\infty} \frac{\sin(x)}{e^x-1} dx = \frac{\pi \coth(\pi)-1}{2}    \href{https://youtu.be/TP7j5kpJeNM?si=n80MSevMN6Fj6FFj}{\,\,=>solution}   \]

\[ \footnote{ Smashing an improper integral using MAZ Identity} \int_0^{\infty} \frac{\sin(ax)}{x^n} dx  \href{https://youtu.be/JE4-212VHf4?si=rQfmOcQ2KzzZ-vbG}{\,\,=>solution}   \]

\[ \footnote{ MAZ Identity speed rockets the integral} \int_1^{\infty} \frac{x^2-1}{x^4 \ln(x)} dx \href{https://youtu.be/K48iKNLhOrg?si=Ra-0BBg67QsnsijB}{\,\,=>solution}   \]

\[ \footnote{ MAZ Identity speed rockets the integral} \int_0^{\infty} \frac{\sin^3(x)}{x^2} dx  \href{https://youtu.be/K48iKNLhOrg?si=VCI3lGgqnF-Y45JX}{\,\,=>solution}   \]

\[ \footnote{ 5 Unusual ways to prove this limit} \lim_{x\to0} \frac{\sin(x)}{x}=1 \href{https://youtu.be/vC4WPqQ8gOs?si=_LBSjcNaPLBFTsCV}{\,\,=>solution}   \]

\[ \footnote{ DIGamma Function helps me solve this infinite sum} \sum_{n=2}^{\infty} \frac{(-1)^n \zeta(n)}{2^n}  \href{https://youtu.be/ZlfG4kWFiVs?si=-8jF8OCQS8QGrUHK}{\,\,=>solution}   \]

\[ \footnote{ Integration of Fraction Part for IIT JEE} \int_1^{\infty} \frac{\{x\}}{x^4} dx = \frac{1}{2}-\frac{\zeta(3)}{3} \href{https://youtu.be/tQ28VbTNqSA?si=TSiHTfFrlkFKN_5W}{\,\,=>solution}    \]

\[ \footnote{ A tricky GIF Integral from MIT Integration Bee} \int_{\frac{1}{4}}^{\frac{1}{2}} \lfloor \log \lfloor \frac{1}{x}\rfloor \rfloor dx \href{https://youtu.be/iZmKik0N58U?si=ofyvjvoq2vw5kLAO}{\,\,=>solution}   \]

\[ \footnote{ MIT Integration BEE Problem that needed MAZ Identity} \int_0^1 \frac{x^7-1}{\log(x)} dx \href{https://youtu.be/pFCHk1iKiIg?si=Emoj8ygpGougrq3k}{\,\,=>solution}   \]

\[ \footnote{ Laplace Transform of \ln(x) } \mathcal{L} \{ \ln(x)\} \href{https://youtu.be/mJKL8ejR30I?si=-kfkSgkm5GGa9-eE}{\,\,=>solution}   \]


\[ \footnote{ Marriage of floor and ceiling function} \int_0^{2022} x^2- \lfloor x \rfloor \lceil x \rceil dx = \frac{2022}{3} \href{https://youtu.be/Kj0XUriy3Qw?si=E1W3EVvI4fWHMvWv}{\,\,=>solution}   \]

\[ \footnote{ A good problem from MIT Integration Bee} \int_0^{\frac{1}{2}} \sum_{n=0}^{\infty} ^{n+3}C_{n} x^n dx  \href{https://youtu.be/2YtegDswySA?si=6OFdaO9OYXpivTPx}{\,\,=>solution}   \]

\[ \footnote{ An Ridiculously Awesome Integral from Ramanujan's land (India) } \int_{0}^{\infty} \frac{\sin(x)}{\sinh(x)} dx = \frac{\pi}{2}\tanh(\frac{\pi}{2}) \href{https://youtu.be/Ts1cEzLifPs?si=HOsWpCpf4LWDmXhh}{\,\,=>solution}   \]

\[ \footnote{ Solving Integrals Geometrically} \int_0^1 \sqrt{1-x^2} dx   \int_1^2 \sqrt{x^2-1} dx \href{https://youtu.be/yPJWZa1p6BQ?si=0JQD4_mMNalRKMBL}{\,\,=>solution}   \]

\[ \footnote{ An Introduction to extremely difficult way to do a simple telescoping sum } \sum_{n=0}^{\infty} \frac{1}{n+2}-\frac{1}{n+3} \href{https://youtu.be/qFyCGAFDBiA?si=d2Rj3Vf2wDynzesa}{\,\,=>solution}  \]

\[ \footnote{ Frullani's Integral} \int_0^{\infty} \frac{f(ax)-f(bx)}{x} dx = (f(\infty)-f(0)) \ln(\frac{a}{b}) \href{https://youtu.be/L3xSB_oADVs?si=9SN30igXi5R-iKID}{\,\,=>solution}  \]

\[ \footnote{ This ridiculously interesting sum is solved by Beta Function} \sum_{n=0}^{\infty} \frac{(n!)^2}{(2n+1)!} = \frac{2\pi}{3\sqrt{3}}} \href{https://youtu.be/2MA5SJlq6JQ?si=rSLmMx3fDtrTVc-S}{\,\,=>solution}  \]

\[ \footnote{ Imaginary Derivative of imaginary number. wow}  \frac{d^i}{dx^i}(x^i) = i!   \href{https://youtu.be/smjSOwctDVk?si=7Oyq18FbpZcmxH4H}{\,\,=>solution}   \]

\[ \footnote{ A brilliant limit from Stanford Maths Tournament} \lim_{n\to\infty} \left( \frac{n!}{n^n} \right) ^{\frac{1}{n}}  = \frac{1}{e}   \href{https://youtu.be/dDJqUkRAC84?si=5z5n_TYULJmFiw_M}{\,\,=>solution}   \]

\[ \footnote{ Differentiation IIT JEE Maths | $\pi$th derivative | Application of Derivative} \frac{d^\pi}{dx^\pi}(x^\pi) = \pi! \href{https://youtu.be/GGfayZj31SY?si=axD061izUNJZTnQU}{\,\,=>solution}  \]

\[ \footnote{ The is the most beautiful problem I ever solved} \lim_{n\to\infty} \sqrt[n]{\Gamma\left(\frac{1}{n}\right)\Gamma\left(\frac{2}{n}\right)\Gamma\left(\frac{3}{n}\right).......\Gamma\left(\frac{n}{n}\right) } \href{https://youtu.be/ildcGfPhgsQ?si=hkbjJsdpZDxjg3Lc}{\,\,=>solution}   \]

\[ \footnote{ Product Integral and Product Derivative} \int f(x)^{dx} , \frac{\delta}{\delta x}f(x)  \href{https://youtu.be/MHjBJASCUH0?si=43joBf4CIkgCliSy}{\,\,=>solution}   \]

\[ \footnote{ Using the powerful stirling approximation for this IIT limit} \lim_{n\to\infty} \left( \frac{n!}{n^n} \right) ^{\frac{1}{n}}  = \frac{1}{e}  \href{https://youtu.be/ZFrMRX0aKr0?si=WQVN3SJAxs2peDSU}{\,\,=>solution}  \]

\[ \footnote{ Proving this IIT Limit using product integral}  \lim_{n\to\infty} \left( \frac{n!}{n^n} \right) ^{\frac{1}{n}}  = \frac{1}{e} \href{https://youtu.be/27OzgS7j7Xg?si=3Z9kGotC8DfXHi3I}{\,\,=>solution}   \]

\[ \footnote{ Infinity factorial, Happy Birthday Bishnu } \infty ! \href{https://youtu.be/bQxzlEp5N0U?si=BcliPF5JGQusRXkI}{\,\,=>solution}  \]


\[ \footnote{ 500 sub special:: Inventing Math: Fractional Derivative, Product Integral and Product Derivative} \frac{d^{\frac{a}{b}}}{dx^{\frac{a}{b}}} f(x) , \int f(x)^{dx} , \frac{\delta}{\delta x}f(x)  \href{write your reference here}{\,\,=>solution}  \]

\[ \footnote{ Impossible seeming Integrals} \int_0^{\frac{\pi}{2}} \sin(x)^{dx} \href{write your reference here}{\,\,=>solution}  \]

\[ \footnote{ Integral with two important constants} \int_0^{1} e^{-x
} \ln^2(x) dx = \frac{\pi^2}{6}+\gamma^2 \href{write your reference here}{\,\,=>solution}   \]

\[ \footnote{Ridiculously Awesome Impossible Integral} \int_0^1 \int_0^1 \int_0^1 \tan^{-1}(xyz) dx dy dz  = - \frac{3\zeta(3)}{32}-\frac{\pi^2}{48}+\frac{\pi}{4}-\frac{\ln(2)}{2} \href{https://youtu.be/1BbA6nCXA5s?si=ZA02cCXXmWs7DVqe}{\,\,=>solution}  \]

\[ \footnote{Ridiculously Awesome Impossible Integral} \int_0^1 \int_0^1 \int_0^1 f(xyz) dx dy dz    = \frac{1}{2} \int_0^1 \ln^2(x)f(x) dx \href{write your reference here}{\,\,=>solution}  \]

\[ \footnote{Ridiculously Awesome Impossible Integral} \int_0^1 \int_0^1  f(xy) dx dy = - \int_0^1 \ln(x)f(x) dx \href{https://youtu.be/PYVXx3U37Sw?si=VFRTvtHgT6fw9zIL}{\,\,=>solution}  \]

\[ \footnote{Ridiculously Awesome Integral} \int_0^1 \int_0^1 \int_0^1 e^{-xyz} dx dy dz  \href{write your reference here}{\,\,=>solution}  \]

\[ \footnote{ A sum from World International Mathematics Olympiad Final 2019} \sum_{k=0}^{10} ^{10}C_k\,\, k^2  = 28160 \href{https://youtu.be/m6z2Jm0Jlhg?si=IPURwPfCM7-3QrCr}{\,\,=>solution} \]

 \footnote{ Suggest} \center {Suggest your favorite integral \\ in the comment for upcoming Video} \href{https://youtu.be/OGVjTKNqB9w?si=fBpaT0zZWGMjeBBs}{\,\,$=>$solution} 
 
\[ \footnote{ The nightmare limit Problem} \lim_{n\to\infty} \frac{n+n^2+n^3+......+n^n}{1^n+2^n+3^n+......+n^n} = 1-\frac{1}{e} \href{https://youtu.be/hmPpH93mWwk?si=QUXNxf8AphF4nPc1}{\,\,=>solution}    \]

\[ \footnote{ Limit involving Prime counting Function} \lim_{n\to\infty} \pi(n) \left( \sqrt[n]{n} -1 \right) = 1 \href{https://youtu.be/SrLb3ch8sDk?si=Xdz5GqqMe50V20qC}{\,\,=>solution}    \]

\[ \footnote{JEE Advanced Limit (Model Question)}  \lim_{n\to\infty} \frac{\lfloor e^{\frac{1}{n}} \rfloor + \lfloor e^{\frac{2}{n}} \rfloor +\lfloor e^{\frac{3}{n}} \rfloor + ....+\lfloor e^{\frac{n}{n}} \rfloor  }{ n }  \href{https://youtu.be/5loCHi2ue-o?si=TihE6c0_OaC1aoEG}{\,\,=>solution}     \]
 
\[ \footnote{ Easy Integral by Himanshu} \int_0^{\infty} \lfloor x \rfloor e^{1-\lfloor x \rfloor } dx  \href{https://youtu.be/D4yyGC7U0RU?si=nTEvPOjicTxme7KH}{\,\,=>solution}     \]

\[ \footnote{ Hard Integral by Himanshu} F(n) = \int_0^{\infty} \frac{x^{2n-2}}{(x^4-x^2+1)^n} dx , F(5)= ?    \href{https://youtu.be/D4yyGC7U0RU?si=nTEvPOjicTxme7KH}{\,\,=>solution}    \]

\[ \footnote{ This is the best use of stirling's approximation} \lim_{n\to\infty} \frac{(2n)! .(2n+1)!}{(n!\,.2^n)^4} \href{https://youtu.be/0JxcugHjrDA?si=wKab5S-c2BChh3lM}{\,\,=>solution}  \]

\[ \footnote{ This is the best use of digamma function} \int_0^1  \int_0^1 \frac{\sqrt{x}+ \sqrt{y}}{\sqrt{\sqrt{xy}}(1-xy)} dx dy  \href{https://youtu.be/1bu8cVyXa5Y?si=L6bVzho5MsmmgMKD}{\,\,=>solution}  \]

\[ \footnote{ Surprise!!!} \int_1^{ \int_1^{\int_1^{\int_1^{\int_1^{\int_1^{.^.^\infty } x dx } x dx} x dx} x dx } x dx}x dx  = 2 + \frac{1}{2+ \frac{1}{2+\frac{1}{2+\frac{1}{2+._._\infty}}}}  \href{https://youtu.be/P7vSE6R_69M?si=tbkxwSlx8CyKHVFK}{\,\,=>solution}  \] 

\[ \footnote { Sum involving King's Rule } \sum_{n=0}^{2023} \frac{1}{5^n + \sqrt{5^{2023}} }  \href{https://youtu.be/N9sx9BvifxQ?si=WgGMrb9xAbGU4aUy}{\,\,=>solution}  \]

\[ \footnote { Destroying five harsh  integrals using Feynman's Technique } \int_0^1 \frac{x-1}{\ln(x)} dx = \ln(2) , \int_0^{\infty} \frac{\sin(x)}{x}= \frac{\pi}{2}, \int_0^1 \frac{\sin(\ln(x))}{\ln(x)} = \frac{\pi}{4} ,\]\[ \int_0^{\infty} \frac{e^{-x^2} \sin(x^2)}{x^2} dx = \sqrt{\pi\sqrt{2}} \sin(\frac{\pi}{8}) , \int_0^{\infty} e^{-x^2} \cos(5x) dx = \frac{\sqrt{\pi}}{2} e^{-\frac{25}{4}} \href{https://youtu.be/F7lp-cgt4CA?si=ERlarP31Dz3Q4FIS}{\,\,=>solution}   \]

\[ \footnote { Integral of Lambert W function} \int W(x) dx \href{https://youtu.be/--mQtsuZI6Q?si=DpUHzn9efKlRGBD1}{\,\,=>solution}  \]
  
\[ \footnote{Can you solve this sum?} \sum_{n=0}^{\infty} \frac{1}{(4n)!} \href{https://youtu.be/fKCOh95srvc?si=onaPnjPcDvdBLXsg}{\,\,=>solution}  \]

\[ min = f(x) = \left( \frac{0.05}{2}e^{ \frac{0.05}{2} ( 2\times 17 + 0.05 \times 20^2-2x) } \times erfc\left(\frac{17+0.05\times20^2-x}{\sqrt{2}\times 20 } \right)\right) \times \frac{3030}{0.0153} \]

\[ \footnote { 6 proofs of Gaussian Integral} \int_0^{\infty} e^{-x^2} dx  \href{https://youtu.be/vt9YpHlYYSw?si=jfe_fkJw2N5Je0AA}{\,\,=>solution} \]

\[ \footnote{ Horse shoe Integral} \int \sin(\sin(x)) \]

\[ \footnote{ MIT would not want to listen this hack about MIT Integration BEE} A difficult integral problem made easy , shorts \]

\[ \footnote { Differentiation of Lambert W function} \frac{d}{dx}W(x) \href{https://youtu.be/hdqHmZqGKd4?si=32WZXG6zv6eyN3ej}{\,\,=>solution}  \] 

\[ \footnote{ Elliptic Integral of the second kind} \int \sqrt{\sin(x)} dx = -2E(\frac{\pi}{4}-\frac{x}{2}\, |\, 2) + c   \href{https://youtu.be/V_E6BXxMOmk?si=slb6e5XhcijmZPfW}{\,\,=>solution}   \]

\[ \footnote{ A nice family of Integrals} \int_0^{\frac{\pi}{2}} \ln(\sin(x)) dx \, \int_0^{\frac{\pi}{2}} \ln(\cos(x)) dx \, \int_0^{\frac{\pi}{2}} \ln(\tan(x)) dx  \href{https://youtu.be/RwGgz9tlaZA?si=uSL9mgYhJQMtaQGB}{\,\,=>solution}   \]
\[ \footnote{ A nice family of Integrals}  \int_0^{\frac{\pi}{4}} \ln(\sin(x)) dx \, \int_0^{\frac{\pi}{4}} \ln(\cos(x)) dx \,\int_0^{\frac{\pi}{4}} \ln(\tan(x)) dx \href{https://youtu.be/RwGgz9tlaZA?si=uSL9mgYhJQMtaQGB}{\,\,=>solution}   \]

\[ \footnote{ One of them is easy and other is hard} \int_0^{\frac{\pi}{4}} \ln(1+\tan(x)) \, \int_0^{\frac{\pi}{4}} \ln(1-\tan(x)) \href{https://www.youtube.com/watch?v=o4TR62QvQJU}{\,\,=>solution}  \]

\[ \footnote { A happy get-together of integrals} \int_0^{\infty} \frac{\ln^2(x)}{1-x^2} dx \,
\int_0^{1} \frac{\ln^2(x)}{1-x^2} dx \, \int_0^{1} \frac{\ln^2(x)}{1+x^2} dx \, \int_0^{\infty} \frac{\ln^2(x)}{1+x^2} dx  \href{https://www.youtube.com/watch?v=pzdGfZ9Gzao}{\,\,=>solution}   \]

\[ \footnote{ Some cool concepts to explain}\] 
1. Proof of Lhopital's rule in both algebraic and visual way \\
2. Fundamental theorem of calculus in a visual way \\
3. Derivative of sin(x) and cos(x) in visual way \\
3. Calculation of Escape Velocity by Newton in 17th Century \\
4. Introduction to Epsilon-Delta Definition \\
5. Zeno's paradox in limits \\
6. What does it mean to be undefined at a point but have limiting value at a point \\
7. Different notation for differentiation of Newton and Leibniz \\
8. Rigorous proof of Euler's Identity from level 0
9. Applications of Differential equation: NEwton's Law of Cooling
10. When to swap the sum and integrals
11. Why does the nth root test work?

\[ \footnote{ Some concepts to use MANIM animation} \] 
1. Cut then multiply method // Demontration in a visual way that (n/x)^x maximizes at e^x i.e. e^(n/e). // Proof using first derivative that it infact is 

2. The reciprocal of the Basel sum answers the question: What is the probability that two numbers selected at random are relatively prime? Excusion in Number Theory Page: 29-35
Citation:  Ogilvy, C. S.; Anderson, J. T. (1988). Excursions in Number Theory. Dover Publications. pp. 29–35. ISBN 0-486-25778-9.

3. 

\[ \footnote { Some videos on  my Checklist} \] 
1. Fractional root of a Matrix, exponential of a matrix, logarithm of a matrix  \\
2. Contour Integration 
3. Usage of Epsilon-Delta Definition and when does it fail ?
4. Deriving Gamma'(1)= - gamma and Gamma''(1) = gamma^2 + zeta(2) and digamma(1/2)=-gamma+2ln(2) 
5. Finding the value of Reimann Zeta of 2 from Level 0 through Digamma Function 


\[ \footnote { A story of two brothers} \int \sqrt{\tan(x)} dx \int \sqrt{\cot(x)} dx  \href{https://www.youtube.com/watch?v=mihN0Vd0IsM}{\,\,=>solution}   \]
 
\[ \footnote { This has a solution!!!} \sin(z)= 2 \href{https://www.youtube.com/watch?v=v1e6oLnQGSk}{\,\,=>solution}     \]

\[ \footnote { A Ridiculously Awesome integral } \int_0^{\infty} \frac{x \cos(x)}{e^x-1} dx    \href{https://www.youtube.com/watch?v=TnwnCS6r00c}{\,\,=>solution}      \]
 
\[ \footnote { A nice Lemma for my nice viewers} \int_0^{\pi} x f(\sin(x)) dx = \frac{\pi}{2} \int_0^{\pi} f(\sin(x)) dx       \href{https://www.youtube.com/watch?v=C1_rrhK5Dnk}{\,\,=>solution}      \]

  
 \[ \footnote { Proof and Usage of Inverse Integration Technique} \int f^{-1}(x) dx = x f^{-1}(x) - F(f^{-1}(x)) + c     \href{https://www.youtube.com/watch?v=-KtP6UAvYZ4}{\,\,=>solution}     \]
 
\[ \footnote{ Proof (algebraic and geometric) and usage of King's Rule} \int_a^b f(x) dx = \int_a^b f(a+b-x) dx    \href{https://www.youtube.com/watch?v=_SqXuNStyXM}{\,\,=>solution}      \]

\[ \footnote { Proof and usage of inverse derivative technique} \frac{d}{dx}( f^{-1}(x)) = \frac{1}{f'(f^{-1}(x))}      \href{https://www.youtube.com/watch?v=GPrmR4iwHSU}{\,\,=>solution}      \]

 
\[ \footnote { Proof(algebraic and geometric)  and usage of definite inverse integration technique} \int_a^b f(x) dx + \int_{f(a)}^{f(b)} f^{-1}(x) dx = bf(b)-af(a)      \href{https://www.youtube.com/watch?v=zl-DunP7Tss}{\,\,=>solution}     \]
 
\[ \footnote{ Verification and usage of Leibniz Rule} (f(x) g(x))^n = \sum_{r=0}^{n} ^nC_r f^r(x)g^{n-r}(x)     \href{https://www.youtube.com/watch?v=5lcBdiT2iXc}{\,\,=>solution}     \] 

\[ \footnote{ Proof and application of Feynman's Technique} \frac{d}{dy}\left( \int_a^b f(x,y) dx \right) = \int_a^b \frac{\partial}{\partial y }(f(x,y)) dx     \href{https://www.youtube.com/watch?v=m9MKHC5UPH4}{\,\,=>solution}     \] 

\[ \footnote { proof and application of complete differentiation using partial differentiation} For f(x,y) = 0 \,  \frac{dy}{dx} = - \frac{f_x}{f_y}     \href{https://www.youtube.com/watch?v=twYmrorc1gc}{\,\,=>solution}    \]

\[ \footnote { proof (algebraic and geometric) and usage of odd/ even function} \int_{-a}^a odd(x) dx  = 0 \, \int_{-a}^a even(x) dx = 2 \int_0^a even(x) dx \href{https://www.youtube.com/watch?v=6aH_t_Iyck8}{\,\,=> Solution}  \]

\[ \footnote { Proof and usage of reflection formula} \int_a^b f(x) dx = - \int_b^a f(x) dx \href{https://www.youtube.com/watch?v=xcoPRjiuTkk}{\,\,=> Solution}  \]

\[ \footnote{ Proof and Usage of DI(Differentiation & Integration) Method} \int f(x)g(x) dx = f \left( \int g \right)  - f' \left( \int\int g \right)  + f'' \left( \int\int\int g \right) -f''' \left(  \int\int\int\int g \right)  + ......  \href{https://www.youtube.com/watch?v=FAY0SWWrr0g}{\,\,=> Solution} \]
  
\[ \int_{-\infty}^{\infty} sech^n(t) dt = \frac{\sqrt{\pi}\, \Gamma (\frac{n}{2})}{\Gamma(\frac{n+1}{2})}  \href{https://www.youtube.com/watch?v=uQ8hogDgz_c}{\,\, => Solution} \]

\[ \footnote { Everything is possible in the realm of complex numbers} 1^z=3 \href{https://www.youtube.com/watch?v=ZeOncjNFoww} {\,\, => Solution} \]

\[ \footnote{ Feynman's Technique is never obvious} \int_0^{\frac{\pi}{2}}\frac{x}{\tan(x)} dx   \href{https://www.youtube.com/watch?v=eX5llnclKQI}{\,\, => Solution} \]
  
 \[ \footnote{ Proof of beta-gamma function using Laplace Tranform and convolution Integral} \beta(m,n)=\frac{\Gamma(m) \Gamma(n)}{\Gamma(m+n)}  \href{https://www.youtube.com/watch?v=SR35oHptwGA} { \,\, => Solution} \] 
 
\[ \footnote{ A simple problem involving Gauss Representation of Gamma Function} \Gamma(z)= \lim_{n\to\infty} \frac{n^z}{z} \prod_{k=1}^{n} \frac{k}{z+k} ||| \frac{\Gamma(x)\Gamma(y)}{\Gamma(x+z)\Gamma(y-z)} =   \prod_{k=0}^{\infty} \left[ \left( 1+\frac{z}{x+k} \right) \left( 1-\frac{z}{y+k} \right) \right]   \href{https://www.youtube.com/watch?v=in0OmqClNNM}{\,\,=> Solution}   \] 
 
\[ \footnote { This is the best use of Legendre's Duplication Formula} \int_0^{\infty} \frac{e^{-t} \cosh(a\sqrt{t}) }{\sqrt{t}} dt \href{https://www.youtube.com/watch?v=bNH2UAP4wHg}{\,\,=> Solution} \]

\[ \footnote { A common sense proof of Legendre's Duplication formula} \Gamma(n+\frac{1}{2})= \frac{(2n)!}{4^n\, n! } \sqrt{\pi}  \href{https://youtu.be/kIwSyds8OTc}{\,\, => solution} \]
 
\[ \footnote { Proving the Euler's Reflection using Sine Product Formula} \Gamma(x) \Gamma(1-x) = \frac{\pi}{\sin(\pi x)} \href{https://www.youtube.com/watch?v=Srd1HAy8dtU} {\,\,=> Solution} \]
 
\[ \footnote { Finding (-1/2)! without gaussian integral} \left( -\frac{1}{2} \right)
!  = \sqrt{\pi} \href{https://www.youtube.com/watch?v=TKNJbZSpahI}{\,\, => Solution} \]

\[ \footnote{ Proving Wallis Product using Sine Product Formula} \frac{2.2}{1.3} . \frac{4.4}{3.5} . \frac{6.6}{5.7}. \frac{8.8}{7.9}.... = \frac{\pi}{2} \href{https://www.youtube.com/watch?v=Wk9FEHsBIGI}{\,\, => Solution} \]

\[ \footnote{ Finding Reimann zeta function of 2 using Sine product formula} \frac{1}{1^2}+\frac{1}{2^2}+\frac{1}{3^2}+\frac{1}{4^2}+ ..... = \frac{\pi^2}{6} \href{https://www.youtube.com/watch?v=J3zCXpZhDjQ}{\,\, => Solution} \]
 
\[ \footnote{ Proving the Sine Product Formula using Digamma Function} \frac{\sin(\pi x)}{\pi x}= \prod_{n=1}^{\infty} \left( 1 - \frac{x^2}{n^2} \right) \href{https://www.youtube.com/watch?v=3vnirncoh1E}{\,\, => Solution}   \]
 
\[ \footnote{ A symmetric Integral }  \int_0^{\frac{\pi}{2}} \ln \left( \sqrt{\sin(x)} + \sqrt{\cos(x)} \right) dx  \]

\[ \footnote{ Finding the value of digamma(1/2)} \psi \left( \frac{1}{2} \right) = -\gamma - 2 \ln(2)  \href{https://www.youtube.com/watch?v=aiFd2nhMiAc}{\,\, => Solution} \]

\[ \footnote{ A simple problem for practice} \int_0^{\infty} \int_0^{\infty} \frac{\tan^{-1}(x^2) \tan^{-1}(y^4)}{x^2 y^3} dx dy  \href{https://www.youtube.com/watch?v=Ttr2h86PbAg}{\,\, => Solution} \]

\[ \footnote { A bonus assignment problem from my mentor} \int_0^1 \int_0^1 \int_0^1 \ln \left( \frac{1}{1+xyz}+ \frac{1}{1-xyz} \right) dx dy dz \href{https://www.youtube.com/watch?v=pajvbI6FiTM}{\,\, => Solution}  \]

\[ \footnote{ This is the best use of polar coordinates} \int_{-\infty}^{\infty} \int_{-\infty}^{\infty} \left( \frac{1}{1+x^2+y^2} \right) ^n dx dy , n \epsilon N , n>1 \href{https://www.youtube.com/watch?v=MMTyYpqGQnA}{\,\, => Solution} \]

\[ \footnote{ When can double factorial be helpful?} \int_0^1 \frac{z^n}{(1-z)^{\frac{1}{2}}} dz = 2. \frac{(2n)!!}{(2n+1)!!} \href{https://www.youtube.com/watch?v=nOk9BVMo95M}{\,\, => Solution} \]
 
 \[ \footnote { Finding digamma (1/2) without using Legendre's Duplication Formula} \psi \left( \frac{1}{2} \right) = - \gamma - 2 \ln(2) \href{https://www.youtube.com/watch?v=XOIY9YEow-U}{\,\, => Solution} \]
 
 
\[ \footnote{ Taylor expansion of \arcsin(x), \arccos(x) and \arctan(x) }  \arcsin(x) = \sum_{n=0}^{\infty} \frac{1}{4^n} \binom{2n}{n} \frac{x^{2n+1}}{2n+1} \href{https://www.youtube.com/watch?v=L4RsZHPGCxY}{\,\, => Solution} \]\[ \arccos(x) = \frac{\pi}{2} - \sum_{n=0}^{\infty} \frac{1}{4^n} \binom{2n}{n} \frac{x^{2n+1}}{2n+1}   \href{https://www.youtube.com/watch?v=L4RsZHPGCxY}{\,\, => Solution} \]\[ \arctan(x) = \sum_{n=0}^{\infty} (-1)^n \frac{x^{2n+1}}{2n+1} \href{https://www.youtube.com/watch?v=L4RsZHPGCxY}{\,\, => Solution} \]
 
\[ \footnote{ A sad story of two brothers} \int_0^{\infty} \frac{x^{n-1}}{1+x}dx =\Gamma(1-n) \Gamma(n)= \frac{\pi}{\sin(n \pi)} \href{https://www.youtube.com/watch?v=_DzVKu6OCwU}{\,\, => Solution} \]
\[ \footnote{ A sad story of two brothers-II} \int_0^{\infty} \frac{x^{n-1}}{1-x}dx =\psi(1-n)-\psi(n) = \frac{\pi}{\tan(n \pi)}  \href{https://www.youtube.com/watch?v=_DzVKu6OCwU}{\,\, => Solution} \] 
% One question to get curious; similar to beta for gamma; what do we have for digamma ?

\[ \footnote{ There's something important to look at} \href{https://en.wikipedia.org/wiki/Taylor_series}{\,\, => \text{ Taylor series in multivariable}} \]

\[ \footnote{ Can you see the factorial in the problem? If yes, this problem is for you } \lim_{k\to\infty} \frac{\sqrt{k}}{e^k} e^{\int_0^{\infty} \lfloor k e^{-x} \rfloor  dx }  \href{https://www.youtube.com/watch?v=PDkpz_nv4fM}{\,\, => Solution} \]

\[ \footnote{ Five fake olympiad problems on youtube and a nice problem} x^{x^3}=729 || 2^x + x = 5 || 2^x=32x || ^2x=16 || 3^x=x^9 || 3^x+3^y+3^z=333  \href{https://www.youtube.com/watch?v=_JIT7nEQYpI}{\,\, => Solution} \]
 
\[ \footnote{ Series Expansion for Gamma Function} \Gamma(1+x) = 1+ \frac{(-\gamma)}{1!}x + \frac{(\gamma^2 +\zeta(2))}{2!}x^2 + O(x^3)  \href{https://www.youtube.com/watch?v=imsmGoekF20}{\,\, => Solution} \]

\[ \footnote{ A nice Problem from Romanian Mathematical Magazine} \int_0^{\infty} \frac{tan^{-1}(x^2)}{1+x^2} + \frac{1}{2} \int_0^{\infty} \frac{tan^{-1}(4x^2)}{1+4x^2} + \frac{1}{3} \int_0^{\infty} \frac{tan^{-1}(9x^2)}{1+9x^2}+..= \frac{\pi^4}{48}    \href{https://www.youtube.com/watch?v=VAD_kttnK_E}{\,\, => Solution}  \]

\[ \footnote{ A beautiful integral for the Euler-Mascheroni Constant} \int_0^1 \left( \frac{1}{\log(x)} + \frac{1}{1-x} \right) dx  = \gamma \href{https://www.youtube.com/watch?v=NTtblQ1fBmQ}{\,\, => Solution} \]

\[ \footnote{ For the love of e } \int_0^{\infty} \frac{\log(x)}{e^x} dx \href{https://youtu.be/AhitFZ31pUM}{\,\, => Solution} \]

\[ \footnote{ How euler found zeta(2) and zeta(4) from Sine product formula?} \zeta(2)= \frac{\pi^2}{6}\,\zeta(4)= \frac{\pi^4}{90} \, \frac{\sin(\pi x)}{\pi x} = \prod_{n=1}^{\infty} \left( 1- \frac{x^2}{n^2} \right) \href{https://www.youtube.com/watch?v=nqz44xlCSgg}{\,\, => Solution} \]

 \[ \footnote { Some fun manipulations on zeta function} \sum_{n=2}^{\infty} \left( \zeta(n) -1 \right)= 1 \,\,  \sum_{n=1}^{\infty} \left( \zeta(2n) -1 \right) = \frac{3}{4}\,\, \sum_{n=1}^{\infty} \left( \zeta(2n+1) -1 \right)  =\frac{1}{4} \href{https://www.youtube.com/watch?v=2u-5Il6fqek}{\,\, => Solution} \]
 
 \[ \footnote{ Something that involves infinite series of digamma and euler-mascheroni constant}   \sum_{n=2}^{\infty} \frac{\zeta(n)-1}{n} = 1- \gamma \]
 
 \[ \footnote{ A sweet introduction to polylogarithm} \ln(x) = \sum_{n=1}^{\infty} (-1)^{n+1} \frac{x^n}{n} || Li_s(z) = \sum_{n=1}^{\infty} \frac{z^n}{n^s} || \zeta(s) = \sum_{n=0}^{\infty} \frac{1}{n^s} || Li_2(z) = Di-Logarithm  \href{https://www.youtube.com/watch?v=NrNXYzQIPsM&t=1s}{\,\, => Solution} \] 


\[ \footnote{ Sum and Integral Representation of Polylogarithm} Li_s(z)=\sum_{n=1}^{\infty} \frac{z^n}{n^s} || Li_s(z)= \frac{1}{\Gamma(s)} \int_0^{\infty} \frac{t^{s-1}}{\frac{e^t}{z}-1} dt  \href{https://youtu.be/GaLpqq-gxRY?si=9UCPUhM7QvB2CdE-}{\,\, => Solution} \]

\[ \footnote{ Properties of Polylogarithm under  Differentiation and Integration} z \frac{\partial (Li_s(z)}{\partial z} = Li_{s-1}(z) || \int_0^z \frac{Li_s(z)}{z} dz = Li_{s+1}(z)   \href{https://youtu.be/fXnA0hCBGCM?si=gn3FCyoEQV-UWBZj}{\,\, => Solution} \]

\[ \footnote{ Reflection Property of Polylogarithm and some more ideas} Li_s(z) + Li_s(-z) = 2^{1-s} Li_s(z^2) \href{https://youtu.be/6TROAHyF6is?si=aCp54mmSFETLKPtq}{\,\, => Solution} \]
  
\[ \footnote{ Two integrals involving Di-logarithm function} \int\frac{tanh^{-1}(x)}{x} dx= \frac{Li_2(x)-Li_2(-x)}{2} || \int_0^1 Li_2(\sqrt{x}) dx = \zeta(2)- \frac{3}{4} \href{https://youtu.be/ILm4t8dss9A?si=F_MkGDoG5uAsjTuM}{\,\, => Solution} \] 

\[ \footnote{ Gauss loves the definition of e } \int_0^1 \frac{1-e^{-x}}{x}dx - \int_1^{\infty} \frac{e^{-x}}{x} dx = \gamma  \href{https://www.youtube.com/watch?v=4JKwKB-Tkb0}{\,\, => Solution} \]

\[ \footnote{ An introduction to Legendre's Chi function} \frac{Li_s(x)-Li_s(-x)}{2}= \chi_s(x) \href{https://youtu.be/m82bKk5JsWM}{\,\, => Solution} \]

% poly-logarithm I want to master pleaseee and contour 
% Cornel Loan Valean // RMM
\[ \footnote{ The u-substitution in this problem is unbelievable . Credit: @nicogehren6556 } \int_0^2 \frac{\ln(1+x)}{x^2-x+1}dx  \href{https://www.youtube.com/watch?v=y_nyEKfKYF8}{\,\, => Solution}  \]

\[ \footnote{ Limit Problem by Cornel Loan Valean on American Mathematical Monthly} \lim_{n\to\infty} \left( \frac{\zeta(2)}{\Gamma(n-2)}  + \frac{\zeta(3)}{\Gamma(n-3)} + .......+\frac{\zeta(n-1)}{\Gamma(1)} \right) \href{https://www.youtube.com/watch?v=2_lMoEm0IpY}{\,\, => Solution}  \]
 
\[ \footnote{ Solving this integral for euler mascheroni constant without digamma function Suggestional Credit: @sigmapoint8333} \int_0^{\infty} \frac{\ln(x)}{e^x} dx  = - \gamma \href{https://www.youtube.com/watch?v=0v_woeO--Uo}{\,\, => Solution}  \]
 
\[ \footnote{ second representation for digamma function credit: Advanced Integration Techniques by Zaid Alyafeai} \psi(z) = \int_0^1 \left( \frac{-1}{\log(t)} - \frac{t^{z-1}}{1-t} \right) dt  \href{https://www.youtube.com/watch?v=uc_-R4qXU4o}{\,\, => Solution}  \]

\[ \footnote{ third representation for digamma function credit: Advanced Integration Techniques by Zaid Alyafeai } \psi(z) = \int_0^{\infty} \left( \frac{e^{-t}}{t} - \frac{e^{-zt}}{1-e^{-t}} \right) dt  \href{https://www.youtube.com/watch?v=svO0WLu6kz0}{\,\, => Solution}  \]

\[ \footnote{ fourth representation of digamma function credit: Advanced Integration Techniques by Zaid Alyafeai} \psi(z) = \int_0^{\infty} \left( \frac{e^{-t}}{t} - \frac{(1+t)^{-z}}{t} \right) dt \href{https://www.youtube.com/watch?v=v1uq2rgnc6g}{\,\, => Solution} \]

\[ \footnote{ Big but easy integral Credit:Advanced Integration Techniques by Zaid Alyafeai } \int_0^1 \frac{(1-x^a)(1-x^b)(1-x^c)}{(1-x)(-\log(x))} dx = \log \left[ \frac{\Gamma(a+b+1) \Gamma(b+c+1) \Gamma(c+a+1)}{\Gamma(a+1) \Gamma(b+1) \Gamma(c+1) \Gamma(a+b+c+1) } \right]  \href{https://www.youtube.com/watch?v=YfAWZl-O1bc}{\,\, => Solution}  \]
 
\[ \footnote{ This integral invokes the fourth integral representation for Di-gamma function Credit:Advanced Integration Techniques by Zaid Alyafeai } \int_0^{\infty} \left( e^{-bx} - \frac{1}{1+ax} \right) \frac{dx}{x} \href{https://www.youtube.com/watch?v=9a8NB5q9Tdo}{\,\, => Solution}  \]
 
\[ \footnote{ An easy problem from Maths Stack Exchange Credit: Guillermo Garc } \int_{-\frac{\pi}{4}}^{0} \prod_{n=0}^{\infty} (1+\tan^{2^n}(x)) dx = \frac{\ln(2)}{4} + \frac{\pi}{8}  \href{https://www.youtube.com/watch?v=HCn4ZvU2dHw}{\,\, => Solution}   \]

\[ \footnote { Let's solve all of these} \int_0^{\infty} e^{-x} dx || \int_0^{\infty} e^{-x^2} dx 
|| \int_0^{\infty} e^{-x^3} dx || \int_0^{\infty} e^{-x^4} dx || \int_0^{\infty} e^{-x^5} dx || \int_0^{\infty} e^{-x^6} dx  || \int_0^{\infty} e^{-x^7} dx  ||\]\[ \int_0^{\infty} e^{-x^8} dx || \int_0^{\infty} e^{-x^9} dx   || \int_0^{\infty} e^{-x^{10}} dx  \href{https://www.youtube.com/watch?v=NMM4iLbGsa8}{\,\, => Solution}  \]

\[ \footnote{ Proof of Stirling's approximation} \lim_{n \to \infty} n! = \sqrt{ 2 \pi n } \left( \frac{n}{e} \right) ^n  \href{https://www.youtube.com/watch?v=8d_hXDoBqEI}{\,\, => Solution}  \]

\[ \footnote{ Proof of the approximation for \Gamma(\frac{1}{n}) } \lim_{n \to \infty} \Gamma( \frac{1}{n})= n - \gamma  \href{https://www.youtube.com/watch?v=NMM4iLbGsa8}{\,\, => Solution}  \]

// To be proved  ( 1 proved) 

\[ \footnote { Proof of the  quadratic series of Au-Yeung} \sum_{n=1}^{\infty} \left( \frac{H_n}{n}\right)^2 \]

\[ \footnote { Proof of the  quadratic series of Au-Yeung} \sum_{n=1}^{\infty} \left( \frac{H_n}{n}\right)^3 \]

\[ \footnote{ Evaluating the variant of the  Integral} \int_0^{\infty} e^{-x^2} dx = \frac{\sqrt{\pi}}{2} || \int_0^{\infty} e^{-ax^2} dx = \frac{1}{2} \sqrt{\frac{\pi}{a}} \]

\[ \int_0^{\infty} e^{-(\ln(x))^2} dx = \sqrt[4]{e}\sqrt{\pi}   || \int_0^{\infty} e^{-(W(x))^2} dx = e^{\frac{1}{4}} \left( \frac{3\sqrt{\pi}}{4} + \frac{e^{-\frac{1}{4}}}{2} + \frac{3\sqrt{\pi}}{4} erf(\frac{-1}{2})  \right)   = 3.0953 \]

\[  \int_0^{\infty} e^{-(H_x)^2} dx = \]


 \[  \int_0^{\infty}  e^{-\Gamma(x)^2} dx = 0.717|| \int_0^{\infty}  e^{-\psi(x)^2} dx || \int_0^{\infty}  e^{-Li_2(x)^2} dx \] 

% Error Function Variant 

\[ \int_0^{\infty}  e^{-erf(x)^2} dx  = DNC|| \int_0^{\infty}  e^{erfc(x)^2} dx =DNC|| \int_0^{\infty}  e^{-erfi(x)^2} dx = 0.728473\]\[ \int_1^{\infty}  e^{-\zeta(x)^2} dx = DNC|| \int_0^{\infty}  e^{-\eta(x)^2} dx || \int_0^{\infty}  e^{-\mathcal{L}(x)^2} dx = DNC\] 

% Trigonometric Variants%

\[  \int_0^{\infty} e^{-(\arcsin(x))^2} dx  = DNC|| \int_0^{\infty} e^{-(\arccos(x))^2} dx = DNC|| || \int_0^{\infty} e^{-(\arctan(x))^2} dx =DNC\]

\[  \int_0^{1} e^{-(\arcsin(x))^2} dx  = \frac{\sqrt{\pi}e^{-\frac{1}{4}}}{4} \left( erfc(\frac{i}{2}) + erfc( \frac{-i}{2}) + i erfi(\frac{1}{2} - \frac{i \pi}{2} ) - i erfi( \frac{1}{2}+\frac{i \pi}{2} ) -2       \right)  \]
\[  \int_0^{1} e^{-(\arccos(x))^2} dx = \frac{\sqrt{\pi} }{4 e^{\frac{1}{4}}} \left(  2 erfi(\frac{1}{2}) - erfi ( \frac{1}{2} - \frac{i \pi}{2}) - erfi( \frac{1}{2} + \frac{i \pi}{2})      \right) \]
\[  \int_0^{1} e^{-(\arctan(x))^2} dx =DNC \]


\[ \int_0^{\infty}  e^{-\sin^2(x)} \int_0^{\infty} e^{-\cos^2(x)} \int_0^{\infty} e^{-\tan^2(x)} 
 \int_0^{\infty}  e^{-\csc^2(x)} \int_0^{\infty} e^{-\sec^2(x)} \int_0^{\infty} e^{-\cot^2(x)} \]

 \[ \int_0^{\frac{\pi}{2}}  e^{-\sin^2(x)} = \frac{\pi}{2 \sqrt{e}} I_0(\frac{1}{2}) \int_0^{\frac{\pi}{2}} e^{-\cos^2(x)} = \frac{\pi}{2 \sqrt{e}} I_0(\frac{1}{2})  \int_0^{\frac{\pi}{2}} e^{-\tan^2(x)}= \frac{e \pi}{2} erfc(1) \]
 \[   \int_0^{\frac{\pi}{2}}  e^{-\csc^2(x)} = \frac{\pi}{2} erfc(1) \int_0^{\frac{\pi}{2}} e^{-\sec^2(x)}= \frac{ \pi}{2} erfc(1) \int_0^{\frac{\pi}{2}} e^{-\cot^2(x)} = \frac{e \pi}{2} erfc(1)\] 
 
 
 % Hyper-Trigonometric Variants%

\[  \int_0^{\infty} e^{-(arcsinh(x))^2} dx  = DNC|| \int_0^{\infty} e^{-(arccosh(x))^2} dx = DNC|| || \int_0^{\infty} e^{-(arctanh(x))^2} dx =DNC\]

\[  \int_0^{1} e^{-(arcsinh(x))^2} dx  = \frac{\sqrt{\pi}}{2} e^{\frac{1}{4}} \]
\[ \int_0^{1} e^{-(arccosh(x))^2} dx = \frac{\sqrt{\pi}}{4} e^{\frac{1}{4}} \left[erf(\frac{1}{2}-\frac{i \pi}{2}) + erf( \frac{1}{2}+ \frac{i \pi}{2}) \right]  \]
\[ \int_0^{1} e^{-(arctanh(x))^2} dx =DNC \]


\[ \int_0^{\infty}  e^{-\sinh^2(x)} \int_0^{\infty} e^{-\cosh^2(x)} \int_0^{\infty} e^{-\tanh^2(x)} 
 \int_0^{\infty}  e^{-csch^2(x)} \int_0^{\infty} e^{-sech^2(x)} \int_0^{\infty} e^{-coth^2(x)} \]

 \[ \int_0^1  e^{-\sinh^2(x)} \int_0^1 e^{-\cosh^2(x)} \int_0^1 e^{-\tanh^2(x)} =   \int_0^1  e^{-csch^2(x)} \int_0^1} e^{-sech^2(x)} \int_0^{1} e^{-coth^2(x)} \] 
 
 \[ \footnote{ Few Gauss-like Integrals}  \int_0^{\infty} e^{-x^2} x^n dx = \frac{1}{2} \Gamma\left(\frac{n+1}{2}\right)     \href{https://youtu.be/PjqznD60wAk}{\,\, => Solution}     \]

\[ \int_0^{\infty} e^{-x^2} \cos(ax) dx = \frac{\sqrt{\pi}}{2} e^{\frac{-a^2}{4}}    \href{https://youtu.be/tAh6Q3YSLHA}{\,\, => Solution}    \]

\[ \int_0^{\infty} e^{-x^2} \sin(ax) dx =  \frac{\sqrt{\pi}}{2} e^{\frac{-a^2}{4}} erfi(\frac{a}{2})     \href{https://youtu.be/vn13Ykz5Lqs}{\,\, => Solution}     \]

\[ \int_0^{\infty} e^{-x^2} \ln(x) dx = -\frac{\sqrt{\pi}}{4} (\gamma + \ln(4))     \href{https://youtu.be/n0Al18iRBaM}{\,\, => Solution}    \]

\[ \int_0^{\infty} e^{-x^2} \cosh(ax) dx = \frac{\sqrt{\pi}}{2} e^{\frac{a^2}{4}}     \href{https://youtu.be/B-QIFtJjx3U}{\,\, => Solution}    \]

\[ \int_0^{\infty} e^{-x^2} \sinh(ax) dx =  \frac{\sqrt{\pi}}{2} e^{\frac{a^2}{4}}erf(\frac{a}{2})     \href{https://youtu.be/euhtnTdgNPY}{\,\, => Solution}    \]

\[ \int_0^{\infty} e^{-x^2} erf(ax) dx =\frac{\arctan(a)}{\sqrt{\pi}}     \href{https://youtu.be/ifJpgF67yZM}{\,\, => Solution}     \]

\[ \int_0^{\infty} e^{-x^2} erfc(ax) dx = \frac{\arctan(\frac{1}{a})}{\sqrt{\pi}}     \href{https://youtu.be/aEDxMTjDR-8}{\,\, => Solution}    \]



\[ \footnote{ Proof of the Classical Euler Sum} \sum_{n=1}^{\infty} \frac{H_n}{n^q}= \frac{(q+2)\zeta(q+1)}{2} - \frac{1}{2} \sum_{k=1}^{q-2} \zeta(k+1) \zeta(q-k) \]

\[ \footnote{ Integral by @sigmapoint8333} \int_0^{\infty} x^2 \frac{\sin(x)}{\sinh(x)} dx \href{https://youtu.be/3xK2woc0_gY?si=75ekbHQbFEvOUpJs}{\,\, => Solution} \]

\[ \footnote{ Abel's Identity for dilogarithm } Li_2 \left( \frac{x}{1-y} \right) + Li_2 \left( \frac{y}{1-x}\right) - Li_2 \left( \frac{xy}{(1-x)(1-y)} \right) = Li_2(x)+ Li_2(y) + \ln(1-x) \ln (1-y) \]

\[ \footnote { Abel Summation} \sum_{k=1}^n a_k b_k = b_{n+1} A_n - \sum_{k=1}^n (b_{k+1}-b_k) A_k  \]\[ \text{ where } A_x = \sum_{i=1}^{x} a_i \]
 


// To be proved 
 
\[ \footnote{ This involves the definition of digamma function Credit:Advanced Integration Techniques by Zaid Alyafeai } \int_0^{\infty} e^{-ax} \left( \frac{1}{x}- \coth(x) \right) dx \href{https://www.youtube.com/watch?v=M47tl6pDUMU}{\,\, => Solution}  \]

\[ \footnote{ How do you find the Bernoulli numbers? }  \frac{t}{e^t -1} }= \sum_{k=0}^{\infty} \frac{B_k}{k!}t^k \]\[ B_0 = 1, B_1 = \frac{-1}{2}, B_2= \frac{1}{6}, B_3= 0 , B_4 = \frac{-1}{30} , B_5 = 0 , B_6= \frac{1}{42}, B_7 = 0, B_8= \frac{-1}{30}, B_9= 0 , B_{10} = \frac{5}{66}\]\[ B_{11}=0 , B_{12} = \frac{-691}{2730}, B_{13}=0, B_{14} = \frac{7}{6}, B_{15}=0 \href{https://www.youtube.com/watch?v=YJNjp6-zT7k}{\,\, => Solution}  \]
 
\[ \footnote{ Come. let me show you the beauty of mathematics} \text{ Prove that odd Bernoulli numbers are zero.}\]\[ B_k : k \ge 3;\text{ k is odd = 0 } Credit: Arizona, planetmath.org \href{https://www.youtube.com/watch?v=Osk7-1VQIdo}{\,\, => Solution}  \] 
 
\[ \footnote{ Proof of this interesting result} B_0= 1 ; \sum_{k=0}^{n} \binom{n+1}{k} B_k = 0 \text{ for }n>0 \href{https://www.youtube.com/watch?v=WBxKp1-ttnA}{\,\, => Solution}  \]
 
\[ \footnote{ the most fascinating use of Bernoulli numbers} 1+2+....+n = \frac{n(n+1)}{2}\]
\[ 1^2+2^2+.......+n^2 = \frac{n(n+1)(2n+1)}{6} \]
\[ 1^3+2^3+.......+n^3 = \frac{n^2 (n+1)^2}{4} \]
\[ 1^4+2^4+.......+n^4 = \frac{n(n+1)(2n+1)(3n^2+3n-1)}{30}\]
\[ 1^5+2^5+.......+n^5 = \frac{n^2(n+1)^2(2n^2+2n-1)}{12} \]
\[ 1^6+2^6+.......+n^6 = \frac{n(n+1)(2n+1)(3n^4+6n^3-3n+1)}{42} \]
\[ 1^k+2^k+.......+n^k = ?   \href{https://www.youtube.com/watch?v=LSNGNcCnqIc}{\,\, => Solution }  \]

\[ \footnote{ Interchanging sum is really beneficial in such cases Credit: An Introduction To The Harmonic Series And Logarithmic Integrals For High School Students Up To Researchers by Ali Shadhar Olaikhan } \sum_{m=1}^{\infty} \sum_{n=1}^{m} x^m \overline{H_n}  \href{https://youtu.be/nFjw7XXw4qg }{\,\, => Solution} \]

\[ \footnote{ Some proofs related to di-logarithm function (also known as Spencer's function) Credit: Maths Stack Exchange , Felix Marin , N3buchadnezzar, Raymond Manzoni } Li_2(z) \href{https://www.youtube.com/playlist?list=PLd4P1gT8vaOORGcoWnX4U5_yoHRmT0tT2 }{\,\, => Solution}   \]
\[\footnote{ Double Identity} Li_2(z) + Li_2(-z) = \frac{1}{2} Li_2(z^2) \href{https://youtu.be/ua6OfdGMnLo }{\,\, => Solution}  \]
\[ \footnote{ Euler's Reflection Formula} Li_2(z)+Li_2(1-z) = \zeta(2)-\ln(z) \ln(1-z) \href{https://youtu.be/wWd8CmUbeVA }{\,\, => Solution}   \]
\[ \footnote{ Landen's Identity} Li_2(-z) + Li_2(\frac{z}{1+z}) = -\frac{1}{2}\log^2(z+1) \href{https://youtu.be/uGcwc1-nMUQ }{\,\, => Solution}    \]
\[ \footnote{Inversion Formula} Li_2(z) + Li_2(\frac{1}{z}) = -\zeta(2)-\frac{1}{2}\log^2(-z)  \href{https://youtu.be/XfKXwMp4k9o }{\,\, => Solution}  \]
 
\[ \footnote{ A cute integral for Apery's constant Credit:Advanced Integration Techniques by Zaid Alyafeai  } \int_0^1 \frac{\log(1-x) \log(x)}{x} dx   \href{https://youtu.be/D72RCn58cQI }{\,\, => Solution}  \]

\[ \footnote{ Let's invoke the tri-gamma function Credit: mathematical reflections, awesomemath.org } \int_0^1 \frac{x \sqrt{x} \ln(x)}{x^2-x+1} dx || \psi'(x) \href{https://youtu.be/-yW1LFyw2lg }{\,\, => Solution}   \]

\[ \footnote{ An easy inequality from Romanian Mathematical Magazine Credit: Daniel Sitaru} \int_a^b \int_a^b \left( \frac{x}{x^4+y^2}+\frac{y}{y^4+x^2} \right)  dx dy \le \ln^2\left( \frac{b}{a} \right) \href{https://youtu.be/75P-TZC7ibo }{\,\, => Solution}   \]

\[ \footnote{ An astonishing integral as a tribute to quad-gamma function Problem Credit: @sigma8333 Solution Credit: Ankush Kumar Parcha} \int_0^{\infty} x^2 \frac{\sin(x)}{\sinh(x)} dx = \frac{\pi^3}{4} \tanh(\frac{\pi}{2}) \text{sech}^2(\frac{\pi}{2})  \href{https://youtu.be/3xK2woc0_gY }{\,\, => Solution}   \]

\[ \footnote{ You will find this integral so cool that you will suffer from cold} \int_0^1 \arcsin(x) \ln(x) dx = 2- \frac{\pi}{2} - \ln(2)   \href{https://youtu.be/yaCXWKgEUos }{\,\, => Solution}  \]

\[ \footnote{ Did you know about this stuff} \frac{d }{d(x)} \left( \beta(x,k) \right) = \beta(x,k) ( \psi(x) - \psi(x+k) )  \href{https://youtu.be/t3aWiKSbYas }{\,\, => Solution}  \]


\[ \footnote{Diagonalising a matrix} \text{ How to write matrix A as } PDP^{-1} \text{ where D is Diagonal Matrix} \text{ and P some other matrix }\]\[  \begin{bmatrix} 2&2 \\ 2& 2 \end{bmatrix} =  \begin{bmatrix} 1& 1 \\ -1 & 1 \end{bmatrix} \begin{bmatrix} 0 & 0 \\ 0& 4 \end{bmatrix} \begin{bmatrix} 1 & 1 \\ -1 & 1 \end{bmatrix}^{-1} \href{https://youtu.be/u0SvENzpxmk }{\,\, => Solution}  \]  

\[ \footnote{ Some property of Matrix Algebra} \text{ If D be a digonal matrix} \begin{bmatrix} a&0 \\ 0&b \end{bmatrix},\text{ and matrix }  A=P D P^{-1} \text{ for some matrix P , then }\]
\[ i) A^n =PD^nP^{-1}= P \begin{bmatrix} a^n&0 \\ 0&b^n \end{bmatrix} P^{-1} \]
 \[ ii) e^A =Pe^DP^{-1}= P \begin{bmatrix} e^a&0 \\ 0&e^b \end{bmatrix} P^{-1}\]
 \[ iii) \ln(A) =P\ln(D)P^{-1}= P \begin{bmatrix} \ln(a)&0 \\ 0&\ln(b) \end{bmatrix} P^{-1} \href{https://youtu.be/MuvTzQjjR1o }{\,\, => Solution}   \]
 
 \[ \footnote{ Square root of a Matrix} \sqrt{ \begin{bmatrix} 2&2 \\ 2&2 \end{bmatrix} }   \href{https://youtu.be/fPeYNxnixMQ }{\,\, => Solution}  \]
 
\[ \footnote{ Exponential of a matrix and some cool insights || Is this just a Miracle? Did this happen by chance} e^{ \begin{bmatrix} 0& -\pi \\ \pi & 0 \end{bmatrix} }  = \begin{bmatrix} -1 & 0 \\ 0 & -1 \end{bmatrix} \href{https://youtu.be/jIMPsvtgJZM }{\,\, => Solution}   \]

\[ \footnote{ Logarithm of a Matrix || Is this just a coincidence??} \ln{ \begin{bmatrix} 0&-1 \\ 1&0\end{bmatrix} = \frac{\pi}{2}\begin{bmatrix} 0 & -1 \\ 1 & 0 \end{bmatrix}   \href{https://youtu.be/768mUkxp-P8 }{\,\, => Solution}  \]

\[ \footnote{ Lambert W of a matrix || This is the most crazy thing on earth} W(\begin{bmatrix} 0 & \pi \\ -\pi & o \end{bmatrix})    \href{https://youtu.be/kY6vlwrQDXw }{\,\, => Solution}   \]

\[ \footnote{ Berkeley Qualifying Exam Question, University of California} \sqrt{ \begin{bmatrix} 1 &3&-3 \\ 0 & 4 & 5 \\ 0 & 0 & 9 \end{bmatrix}    \href{}{\,\, => Solution}  \]
\[ \footnote{ Let's do this under few second} \begin{bmatrix} 1&0 \\ 0 & 3 \end{bmatrix} ^ {35}  \href{https://youtu.be/_sS6W82xmuo }{\,\, => Solution}    \]

\[ \footnote{ Matrix raised to a Matrix} \begin{bmatrix} 0&1 \\ 1&0 \end{bmatrix} ^ { \begin{bmatrix} 1&-1 \\ -1 & 1 \end{bmatrix} }    \href{}{\,\, => Solution}  \]

\[ \footnote{ Matrixth root of a Matrix} \sqrt[ \begin{bmatrix} 2 & -1 \\ -3 & 2 \end{bmatrix}]{ \begin{bmatrix} 1&0 \\ -3 & 2\end{bmatrix}} || \begin{bmatrix} 1&0 \\ -3 & 2\end{bmatrix}^{\frac{\Big1}{\begin{bmatrix} 2 & -1 \\ -3 & 2 \end{bmatrix}}}  \href{}{\,\, => Solution}   \]

 
\[ \footnote{ Finding all of these zeta values} \zeta(2)= \frac{\pi^2}{6} || \zeta(4)= \frac{\pi^4}{90} || \zeta(6) = \frac{\pi^6}{945} || \zeta(8)= \frac{\pi^8}{9450} || \zeta(10)= \frac{\pi^{10}}{93555} \] \[ \zeta(12) || \zeta(14) || \zeta(16) || \zeta(18) || \zeta(20) || \zeta(22) || \zeta(24) || \zeta(26) || \zeta(28) || \zeta(30)   \href{https://youtu.be/CO_VUEQ4xAA }{\,\, => Solution}   \]

\[ \footnote{ Relation between Hurwitz zeta and poly-gamma function} \zeta(s,a)= \frac{\psi_{s-1}(a)}{(-1)^s(s-1)!}  \href{https://youtu.be/65wPYSdBJyw }{\,\, => Solution}   \]


\[ \footnote{ Let's get adapted with Di-logarithm} \int_0^x \frac{\ln^2(1-t)}{t} dt , 0<x<1  \href{https://youtu.be/UbToSM354_0?si=y0Li97d9NHhOQGzH }{\,\, => Solution}   \]

\[ \footnote{ Monstrous but easy integral} \int_0^1 \int_0^1 \int_0^1 \frac{x^2y^2z^2 \ln(xyz)}{1-x^2y^2z^2} dx dy dz = -\frac{\pi^4}{32}+3   \href{https://youtu.be/2ThTO-k-KvI?si=YFzoEl6_eYZabfcT  }{\,\, => Solution}   \]

\[ \footnote{ Everyone can solve the first one. Can you solve the second one?} \int_0^{\infty} \frac{x}{e^x-1} dx || \int_0^a \frac{x}{e^x -1 } dx   \href{https://youtu.be/TPtH4tmkj68?si=aKOHQG1OXiaH1eRW}{\,\, => Solution}     \]

\[ \footnote{ A short introduction to hyper geometric function} _2F_1(a,b;c;z)=\sum_{n=0}^{\infty} \frac{(a)_n(b)_n}{(c)_n} \frac{z^n}{n!} \href{https://youtu.be/m0_AoyqFhFQ?si=E3yio-ZYN7i8YITQ}{\,\, => Solution}      \] where (k)_n=k(k+1)..(k+n-1) 


\[ \footnote{ Representation of some famous functions using hyoer-geometric functions} \text{Logarithm} \ln(1+z)=  _2F_1(1,1;2;-z)z   \href{https://youtu.be/XiU1dauB5cQ?si=Wsn9QN71FkorvVhw}{\,\, => Solution}   \]
\[ \text{ Power Function } (1-z)^{-a}= _2F_1(a,1;1;z)  \href{https://youtu.be/XiU1dauB5cQ?si=Wsn9QN71FkorvVhw}{\,\, => Solution}   \]
\[ \text{ Arcsin Function } \arcsin(x)= _2F_1(\frac{1}{2},\frac{1}{2};\frac{3}{2};z^2)z \href{https://youtu.be/XiU1dauB5cQ?si=Wsn9QN71FkorvVhw}{\,\, => Solution}   \]
\[ \text{ Geometric Series } (1-z)^{-1}=_2F_1(1,1;1;z) \href{https://youtu.be/XiU1dauB5cQ?si=Wsn9QN71FkorvVhw}{\,\, => Solution}   \]
\[ \text{ Exponential Function } e^z = _2F_1(-,-;-;z) \href{https://youtu.be/XiU1dauB5cQ?si=Wsn9QN71FkorvVhw}{\,\, => Solution}   \]
\[ \text{ Sine Function } \sin(z)= _2F_1(-,-;\frac{3}{2};\frac{-z^2}{4})z \href{https://youtu.be/XiU1dauB5cQ?si=Wsn9QN71FkorvVhw}{\,\, => Solution}    \]
\[ \text{ Cosine Function } \cos(z)=_2F_1(-,-;\frac{1}{2};\frac{-z^2}{4}) \href{https://youtu.be/XiU1dauB5cQ?si=Wsn9QN71FkorvVhw}{\,\, => Solution}    \]
 
\[ \footnote{ You can solve the first integral. But can you solve the second integral? An application of Hyper-Geometric Function} \int_0^1 x^{-\frac{1}{2}} (1-x)^{-\frac{1}{4}} dx || \int x^{-\frac{1}{2}} (1-x)^{-\frac{1}{4}} dx    \href{https://youtu.be/-3sAJ2Wr9YM?si=3bts0BSEX6XUKiZ9}{\,\, => Solution}   \]

\[ \footnote{ Proving the Integral Representation for Hyper-Geometric function} \beta(c-b,b) _2F_1(a,b;c;z)= \int_0^1 t^{b-1} (1-t)^{c-b-1} (1-tz)^{-a} dt    \href{https://youtu.be/fwlbULTeNY4?si=r8iIVlPbl6EUJCtt}{\,\, => Solution}    \]

\[ \footnote{ Proof of the Pfaff tranformation of Hyper-geometric function} _2F_1(a,b;c;z)=(1-z)^{-a} \,\, _2F_1(a,c-b;c;\frac{z}{z-1}) = (1-z)^{-b}\,\, _2F_1(c-a,b;c;\frac{z}{z-1})   \href{https://youtu.be/_cCec8DCQ6M?si=JptCxOr3YvpO9X9v}{\,\, => Solution}     \]

\[ \footnote{ Proof of the Euler Transformation of Hyper-geometric function} _2F_1(a,b;c;z)=(1-z)^{c-a-b} \,\, _2F_1(c-a,c-b;c;z)   \href{https://youtu.be/yvprsLOx7Ko?si=xXn2p8ALT0FzhcHd}{\,\, => Solution}     \]

\[ \footnote{ Some Special Values of Hyper-geometric function at 1} _2F_1(a,b;c;1)=\frac{\Gamma(c) \Gamma(c-a-b)}{\Gamma(c-a) \Gamma(c-b)} \href{https://youtu.be/QzMmmn7bNZ8?si=ewwN3d0KEXjySWcI}{\,\, => Solution}      \]

\[ \footnote{ Some special values of Hyper-geometric function at -1} _2F_1(a,b;1+a-b;-1)=\frac{\Gamma(1+a-b) \Gamma(1+\frac{a}{2})}{\Gamma(1+\frac{a}{2}-b) \Gamma(1+a)} \href{https://youtu.be/QzMmmn7bNZ8?si=ewwN3d0KEXjySWcI}{\,\, => Solution}     \]

\[ \footnote{ Introduction and relation between these function} \frac{\sqrt{\pi}}{2}erf(x)=\int_0^x e^{-t^2} dt || \frac{\sqrt{\pi}}{2} erfc(x)=\int_x^{\infty} e^{-t^2} dt || \frac{\sqrt{\pi}}{2} erfi(x) = \int_0^x e^{t^2} dt \href{https://youtu.be/Eo7OjMotpxU?si=iCs0HRwFbfWPby_X}{\,\, => Solution}    \]

\[ \footnote{ Relation of error function with hypergeometric functions and incomplete beta function} erf(x)=\frac{2x}{\sqrt{\pi}}\, _2F_1 \left(-, \frac{1}{2};\frac{3}{2};-x^2} \right) || erf(x)=1-\frac{\Gamma(\frac{1}{2},x^2)}{\sqrt{\pi}}     \href{https://youtu.be/2UP1aOK1gF0?si=PpJpfS1xm3DNbQy9}{\,\, => Solution}    \]

\[ \footnote{ Trig+error=Trigger function} \int_0^{\infty} \sin(x^2) erfc(x) dx = \frac{\pi-2\coth^{-1}(\sqrt{2})}{4\sqrt{2}  \pi}  \href{https://youtu.be/Asr1abPKEYI?si=WiCK4Xjy941bpSVz}{\,\, => Solution}    \]
 
\[ \footnote{ A somehow unpopular technique} \int_0^{\infty} erfc(x) e^{-2x^2} dx  \href{https://youtu.be/d3dVBvRDpV8?si=lEyjDe2ovnK1bG4N}{\,\, => Solution}    \]
 
\[ \footnote{ How far can we go?} \int_0^{\infty} erfc(x) dx || \int_0^{\infty} erfc^2(x) dx || \int_0^{\infty} erfc^3(x) dx .....     \href{https://youtu.be/vUWoLhlj0Oo?si=9vN6kh4tnj6AoG6A}{\,\, => Solution}    \]
 
\[ \footnote{ Modified Gaussian Integral-I} \int_0^{\infty} e^{-\ln^2(x)} dx = \sqrt[4]{e} \sqrt{\pi}      \href{https://youtu.be/9E9uvgIy0tc?si=n8PidKkP3rjfCyyh}{\,\, => Solution}     \]

\[ \footnote{ Modified Gaussian Integral - II} \int_0^{\infty} e^{-W(x)^2} dx = e^{\frac{1}{4}} \left[ \frac{3\sqrt{\pi}}{4}+ \frac{e^{\frac{-1}{4}}}{2} - \frac{3\sqrt{\pi}}{4}erf\left(\frac{-1}{2}\right) \right]     \href{https://youtu.be/xyZnCyIjfDg?si=nsVQRSm9tMMuImFf}{\,\, => Solution}     \]
 
\[ \footnote{ Sum and Integral Representation for Exponential Integral Function} E(z)= \int_z^{\infty} \frac{e^{-t}}{t} dt = \int_1^{\infty} \frac{e^{-zt}}{t} dt    \href{https://youtu.be/fV0NxZFfSis?si=8B8b4M3q5nONYf9n}{\,\, => Solution}     \]
\[ E(z)= -\gamma - \ln(z) + \int_0^z \frac{1-e^{-u}}{u} du ||  E(z)= - \gamma - \ln(z)+ \sum_{k=1}^{\infty} \frac{(-1)^{k+1} z^k}{k!k}     \href{https://youtu.be/fV0NxZFfSis?si=8B8b4M3q5nONYf9n}{\,\, => Solution}     \]

\[ \footnote{ A nice limit problem for Exponential Integral Function Credit: Advanced Integration Techniques by Zaid Alyafeai} \lim_{x\to0} \left[ \log(x) + E(x) \right] = - \gamma      \href{https://youtu.be/jy3ohA4gZwk?si=GoQwdhNQNHGYmJ-C}{\,\, => Solution}     \]

\[\footnote{Try this simple integral involving Exponential Integral Function Credit: Advanced Integration Techniques by Zaid Alyafeai} \int_0^{\infty} x^{p-1} E(ax) dx = \frac{\Gamma(p)}{pa^p}      \href{https://youtu.be/gPc2fuJWR1Q?si=u_V7cKLW5q5TcwuO}{\,\, => Solution}     \]

\[ \footnote{ This is a good problem to review Exponential Integral Function Credit: Advanced Integration Techniques by Zaid Alyafeai} \int_0^{\infty} x^{p-1} e^{ax} E(ax) dx = \frac{\pi}{\sin(p\pi)} . \frac{\Gamma(p)}{a^p}       \href{https://youtu.be/7Fy4mCjCnIs?si=GkmFpRPRGxAWIE6z}{\,\, => Solution}     \]


 \[ \footnote{ Surprise!!!!} \int_0^{\infty} e^z E^2(z) dz = \zeta(2)     \href{https://youtu.be/cMbBvbAN-d8?si=X8ktD8UcGAAlrnXa}{\,\, => Solution}     \]
 
\[ \footnote{ This is a regular standard boring integral, but still I am doing it. Why? } \int_0^1 \frac{x \ln^2(x)}{x^3+x\sqrt{x}+1} dx = \frac{8}{729} \left( \psi''(\frac{7}{9})- \psi''(\frac{4}{9}) \right)     \href{https://youtu.be/3WUa62oShZg?si=SbGYu-8CEJ2ON4Z3}{\,\, => Solution}     \]
 
\[ \footnote{ Jensen Inequality problem from School of Science and Math Journal November 2007 (school science and math journal) }        \]
\textit{ 4970  : Proposed by Isabel Dıaz-Iriberri and Jose Luis Dıaz-Barrero, Barcelona, Spain.}\\
Let f : [0, 1] → R be a continuous convex function. Prove that:  
\[ \frac{3}{4} \int_0^{\frac{1}{5}} f(t) dt + \frac{1}{8} \int_0^{\frac{2}{5}} f(t) dt \ge \frac{4}{5} \int_0^{\frac{1}{4}} f(t) dt    \href{https://youtu.be/FgNr0c_Wjoo?si=bM20BBJUOSpA-qNK}{\,\, => Solution}    \]
 
\[ \footnote{ You might enjoy this integral (school science and math journal) }\]
\textit{ 4983 : Proposed by Ovidiu Furdui, Kalamazoo, MI.} \\
Let k be a positive integer. Evaluate: 
\[ \int_0^1 \Big\{ \frac{k}{x} \Big\} dx   \href{https://youtu.be/bbme4UDYwRI?si=vVDumeRT649QKGjS}{\,\, => Solution}    
\]
where \{a\} is the fraction part of a. 

\[ \footnote{ A simple sum from Binomial Expansion (school science and math journal) } \]
\textit{ 4996 : Proposed by Kenneth Korbin, New York, NY } \\
Simplify: 
\[\sum_{i=1}^N \binom{N}{i}(2^{i-1})(1+3^{N-i}) = \frac{5^N-1}{2}   \href{https://youtu.be/NAZlL_fC9O8?si=puU_LIVpWnzGXUQf}{\,\, => Solution}     \]

\[ \footnote{ Do you see Wallis Product in this sum (school science and math journal)} \]
\textit{ • 5006  : Proposed by Ovidiu Furdui, Toledo, OH } \\
Find the sum :
\[ \sum_{k=2}^{\infty} (-1)^k \ln\left(1-\frac{1}{k^2}\right) = \ln\left(\frac{8}{\pi^2}\right)  \href{https://youtu.be/UQTo3X6D_xo?si=dye6noQEI6KaXjUx}{\,\, => Solution}    \]

\[ \footnote{ This problem involves Ramanujan's nested Radical (school science and math journal)} \]
\textit{ • 5068  : Proposed by Kenneth Korbin, New York, NY } \\
Find the value of \\
\[  \sqrt{1+2009\sqrt{1+2010\sqrt{1+2011\sqrt{1+....}} }}  \href{https://youtu.be/yGpJmDnlv7Y?si=4HxYyBTvxZCrx7rf}{\,\, => Solution}   
\]

\[ \footnote{ Integrals involving fractional part function are so beautiful} \]
\textit{ • 5073: Proposed by Ovidiu Furdui, Campia-Turzii, Cluj, Romania. } \\
Let m $>$ - 1 be a real number. Evaluate:
\[ \int_0^1 \{ \ln(x) \} x^m dx       \href{https://youtu.be/fUI8zS5SRp8?si=SluEzlCEqr3duLja}{\,\, => Solution}    \]
where \{a\} = a - [a] denotes the fractional part of a. 


\[ \footnote{ The answer is irrational fraction} \int_0^1 \{ - \ln(x) \} dx \]

\[ \footnote{ You have to take care of this integral geometrically} \int_0^1 \int_0^1 \{ y^2-x\} dx dy = \frac{1}{2}    \href{https://youtu.be/vYczMyh4MUM?si=2CZAPTuMDOTqoNEM}{\,\, => Solution}    \]


\[ \footnote{ Ramanujan's nested radical returns} \]
\textit{ • 5118: Proposed by David E. Manes, Oneonta, NY} \\
Find the value of :
\[ \sqrt{2011+ 2007 \sqrt{2012+ 2008 \sqrt{ 2013+ 2009 \sqrt{ 2014+.....}} }}= 2009   \href{https://youtu.be/yGpJmDnlv7Y?si=RAc9DlhOFJ9YW6Lg}{\,\, => Solution}   
\]
\footnote{ Similar ramanujan's problem created by myself} 
\[ \sqrt{1^2+\sqrt{2^2+\sqrt{4^2+\sqrt{8^2+\sqrt{16^2+.....}}}}} = 2    \href{solution video}{\,\, => Solution}    \]

\[ \sqrt{1+\sqrt{1+2^2+\sqrt{2+3^2+\sqrt{3+4^2+\sqrt{4+5^2+....}}}}}= 2  \href{solution video}{\,\, => Solution}    \]

\[ \sqrt{1^2+\sqrt{2^2+\sqrt{3^2+\sqrt{4^2+\sqrt{5^2+.....}}}}} = ?   \href{solution video}{\,\, => Solution}    \]

\[ \footnote{ The math never lies}\]
\textit{ • 5139: Proposed by Ovidiu Furdui, Cluj, Romania } \\
Prove: \\
\[ \sum_{n=1}^{\infty} \sum_{m=1}^{\infty} \frac{\zeta(m+n)-1}{m+n} = \gamma ,   \href{https://youtu.be/AMpkY16EBN4}{\,\, => Solution}   \]
 where $\zeta$ denotes the Riemann zeta function. 


\[ \footnote{ A pretty standard and easy problem to know} \]
\textit{• 5174: Proposed by Jose Luis Dıaz-Barrero, Barcelona, Spain } \\
Let n be a positive integer. Compute: \\
\[ \lim_{n \to \infty} \frac{n^2}{2^n} \sum_{k=0}^n \frac{(k+4)}{(k+1)(k+2)(k+3)} \binom{n}{k}  \href{https://youtu.be/RYXKUO1LNqo}{\,\, => Solution}     
\]

\[ \footnote{This sum involves Riemann Definition of double integral} \]
\textit{ • 5175: Proposed by Ovidiu Furdui, Cluj-Napoca, Romania } \\
Find the value of: \\
\[ \lim_{n \to \infty} \frac{1}{n} \sum_{i,j=1}^{n} \frac{i+j}{i^2+j^2}  \href{https://youtu.be/SCaXn3PEoRw}{\,\, => Solution}   
\]

\[ \footnote{ Feels like using beta function? } \]
\textit{ • 5181: Proposed by Ovidiu Furdui, Cluj, Romania} \\
Calculate: \\
\[ \sum_{n=1}^{\infty} \sum_{m=1}^{\infty} \frac{n.m}{(m+n)!}  \href{https://youtu.be/R9PfetRWHJ8}{\,\, => Solution}    \]


\[ \footnote{ Happy Pi day guys}  \int_0^1 \frac{x^4 (1-x)^4}{1+x^2} dx = \frac{22}{7}-\pi   \href{https://youtu.be/V_5OypB3Ngs}{\,\, => Solution}   \]

\[ \footnote{ Deriving the area of circle using l'hopital's rule} \text{Area of Circle}= \pi r^2  \href{https://youtu.be/VIYuir2ecZ4}{\,\, => Solution}    \]

\[ \footnote{ You can solve the first integral but can you solve the second one?} \int_0^{\frac{\pi}{2}} \ln(\sin(x)) dx = -\frac{\pi}{2}\ln(2) \]
\[ \int \ln(\sin(x)) dx = ?    \href{https://youtu.be/hnMvpDiQWV0}{\,\, => Solution}    \]

\[ \footnote{ What an extra-ordinary result?} \int_{-\infty}^{\infty} \binom{n}{x} dx = \sum_{x=0}^{\infty} \binom{n}{x}   \href{https://youtu.be/mpIvTl9auWk}{\,\, => Solution}    \]

\[ \footnote{ This Cambridge integral involves Glasser's Master Theorem} \int_0^{\infty} e^{-c (y+y^{-1})} y^{-\frac{1}{2}} dy    \href{https://youtu.be/M6IrEge3zw0}{\,\, => Solution}    \]

\[ \footnote{ I bet you got this wrong} \int_{-\infty}^{\infty} \frac{\sin(x-\frac{1}{x})}{x-\frac{1}{x}} (1+\frac{1}{x^2}) dx \neq \pi\text{ but }= 2\pi   \href{https://youtu.be/bxa10CAceBQ}{\,\, => Solution}    \]

\[ \footnote{ Grand Glasser's Master theorem is so powerful} \int_0^{\infty} sech^2(x+\tan(x)) dx \href{https://youtu.be/o_uvftIQqpA}{\,\, => Solution}     \]

\[ \footnote{ What a beautiful answer to have?} \int_{-\infty}^{\infty} \frac{2x^2}{x^4+2x^2+5} dx = \frac{\pi}{\sqrt{\phi}} \href{https://youtu.be/4xqs6IVXDeE}{\,\, => Solution}    \]

\[ \footnote{ Proof of this amazing series} \cot(x)= \sum_{k \in z} \frac{1}{x+k \pi}  \href{https://youtu.be/_z3-FdJxNOg}{\,\, => Solution}   \]
\[ cosec(x)= \sum_{ k \in z} \frac{(-1)^k}{x+k \pi }    \href{solution video}{\,\, => Solution}   \]

\[ \footnote{ Modified Gaussian Integral- III} \int_0^{\frac{\pi}{2}} e^{-tan^2(x)} dx  \href{https://youtu.be/oX8oGRaF32I}{\,\, => Solution}    \]

\[ \href{https://mathworld.wolfram.com/topics/SpecialFunctions.html}{special functions}  \href{solution video}{\,\, => Solution}    \]


\[ \footnote{ Modified Gaussian Integral - IV } \int_0^{\frac{\pi}{2}} e^{-\cot^2(x)} dx    \href{https://youtu.be/nbWTBkPzl5k}{\,\, => Solution}   \]

\[ \footnote{ Modified Gaussian Integral - V } \int_0^{\frac{\pi}{2}} e^{-\sec^2(x) dx      \href{https://youtu.be/jcAa4hXZb30}{\,\, => Solution}    \]

\[ \footnote{ Modified Gaussian Integral - VI} \int_0^{\frac{\pi}{2}} e^{-\csc^2(x)} dx      \href{https://youtu.be/NFvlfx5egAc}{\,\, => Solution}    \]

\[ \footnote{ Modified Gaussian Integral - VII} \int_0^{\frac{\pi}{2}} e^{-\sin^2(x)} dx   \href{https://youtu.be/UWmtOH_SPOc}{\,\, => Solution}    \]
 
\[ \footnote{ Modified Gaussian Integral - VIII} \int_0^{\frac{\pi}{2}} e^{-\cos^2(x)} dx    \href{https://youtu.be/onVmi4NAHNY}{\,\, => Solution}    \]

\[ \footnote{ Modified Gaussian Integral - IX } \int_0^1 e^{-\arcsin^2(x)} dx    \href{https://youtu.be/u6eCE4HnUCc}{\,\, => Solution}    \]
 
\[ \footnote{ Modified Gaussian Integral - X } \int_0^1 e^{ - \arccos^2(x)} dx    \href{https://youtu.be/NSHV4qbOoOI}{\,\, => Solution}    \]

\[ \footnote{ An introduction to Complete Elliptic Integral of 1st kind and 2nd kind } 
K(k)= \int_0^{\frac{\pi}{2}} \frac{d\theta}{\sqrt{1-k^2 \sin^2(\theta)} } = \int_0^{1} \frac{dx}{\sqrt{1-k^2 x^2} \sqrt{1-x^2} }   \href{https://youtu.be/EOr4FUp4ZbI}{\,\, => Solution}     \]
 
\[  E(k)= \int_0^{\frac{\pi}{2}} \sqrt{1-k^2 \sin^2(\theta)} d\theta = \int_0^{1} \frac{\sqrt{1-k^2 x^2}}{ \sqrt{1-x^2} }dx     \href{https://youtu.be/EOr4FUp4ZbI}{\,\, => Solution}   \]

\[ \footnote{ Deriving the Hyper-Geometric Series for Complete Elliptic Integrals of First and second kind using both sum and integral definition of hyper-geometric function} K(k)=\frac{\pi}{2}\, _2F_1\left( \frac{1}{2},\frac{1}{2};1;k^2 \right)      \href{https://youtu.be/02ny5oA8ZxM}{\,\, => Solution}     \]

\[  E(k)=\frac{\pi}{2}\, _2F_1\left( \frac{1}{2},-\frac{1}{2};1;k^2 \right)     \href{https://youtu.be/02ny5oA8ZxM}{\,\, => Solution}    \]

\[ \footnote{ Modified Gaussian Integral - XI // A Beautiful. Crazy Girlfriend of Gaussian Integral} \int_0^{\infty} e^{-arcsinh(x)^2} dx = \frac{\sqrt{\pi}}{2} e^{\frac{1}{4}}    \href{https://youtu.be/lXKD6SzNS4c}{\,\, => Solution}     \]

\[ \footnote{ Modified Gaussian Integral - XII } \int_0^{\infty} e^{-arccosh(x)^2} dx      \href{https://youtu.be/XYHn9eU7FL0}{\,\, => Solution}     \]
 
\[ \footnote{ The celebrity return of Catalan's Constant} \int_0^1 K(k) dk = 2 G     \href{https://youtu.be/9Q-iTryChek}{\,\, => Solution}     \]
where G is the catalan's constant and K(k) is complete Elliptic Integral of first kind.

\[ \footnote{ Proof of the identities involving Complete Elliptic Integral of First and Second Kind} K \left( \sqrt{\frac{k}{k-1}} \right) =  K(\sqrt{k}) \sqrt{1-k}      \href{https://youtu.be/lzr-IwmPc-U}{\,\, => Solution}      \]

\[  E \left( \sqrt{\frac{k}{k-1}} \right) = \frac{ E(\sqrt{k})}{\sqrt{1-k}}   \href{https://youtu.be/lzr-IwmPc-U}{\,\, => Solution}      \]

\[ \footnote{ Proof of the identities involving Complete Elliptic Integral of First and Second Kind - II} K \left( \frac{2\sqrt{k}}{1+k} \right) =  \frac{1+k}{1-k} K \left( \frac{2\sqrt{-k}}{ 1-k} \right)      \href{https://youtu.be/1ak5ebINqFE}{\,\, => Solution}      \]

\[ E \left( \frac{2\sqrt{k}}{1+k} \right) =  \frac{1-k}{1+k} E \left( \frac{2\sqrt{-k}}{ 1-k} \right)    \href{https://youtu.be/1ak5ebINqFE}{\,\, => Solution}     \]

\[ \footnote{ Special Values of Complete Elliptic Integal of first and second Kind} K(i)= \frac{1}{4\sqrt{2\pi}} \Gamma^2(\frac{1}{4})  \href{https://youtu.be/bGP8Ayfxrxo}{\,\, => Solution}      \]

\[ E(i)= \frac{1}{4 \sqrt{2\pi}} \Gamma^2(\frac{1}{4}) + \frac{1}{\sqrt{2 \pi}} \Gamma^2(\frac{3}{4})    \href{https://youtu.be/bGP8Ayfxrxo}{\,\, => Solution}     \]

\[ \footnote{ Special Values of Complete Elliptic Integral of first and second kind} K(\frac{1}{\sqrt{2}}) = \frac{1}{4\sqrt{\pi}} \Gamma^2(\frac{1}{4})   \href{https://youtu.be/hwu25FmziKw}{\,\, => Solution}    \]

\[ E(\frac{1}{\sqrt{2}})= \frac{1}{8 \sqrt{\pi}} \Gamma^2(\frac{1}{4}) + \frac{1}{2\sqrt{ \pi}} \Gamma^2(\frac{3}{4})     \href{https://youtu.be/hwu25FmziKw}{\,\, => Solution}     \]

\[ \footnote{ The Legendary Derivatives of Elliptic Integrals} \frac{d}{dk}(E(k))= \frac{1}{k} \left[ E(k)-K(k) \right]       \href{https://youtu.be/KhERmSxBoVY}{\,\, => Solution}      \]

\[ \frac{d}{dk}(K(k))= \frac{1}{k} \left[ \frac{E(k)}{1-k^2}-K(k) \right]      \href{https://youtu.be/KhERmSxBoVY}{\,\, => Solution}     \]
 

\[ \footnote{ Come, let me prove my claim} \zeta(0)= -\frac{1}{2}      \href{https://youtu.be/GXfn4rqWgQQ}{\,\, => Solution}     \]
  
  
\[ \footnote{ Using green's theorem to derive area of circle formula}  A = \pi r^2     \href{https://youtu.be/KQgycv2UGj0}{\,\, => Solution}     \]
 
\[ \footnote{ Try this easy Romanian College entrance exam problem} \int (x^6+x^3) \sqrt[3]{x^3+2} dx    \href{https://youtu.be/ibIUiEgrS-s}{\,\, => Solution}     \]

\[ \footnote{ A Clever u-substitution Credit: @mathematician6124 } \int \frac{1}{(1-x^2)\sqrt[4]{2x^2-1}} dx     \href{https://youtu.be/1Vfls-Tr6Ro}{\,\, => Solution}     \]
 
\[ \footnote{ A cutie integral} \int_a^b \frac{e^{\frac{x}{a}}-e^{\frac{b}{x}}}{x} dx    \href{https://youtu.be/Gteqrba0JDc}{\,\, => Solution}    \] 
 
\[ \footnote{ Guys, I crafted an amazing solution for this integral} \int_0^{\frac{\pi}{2}} \frac{x\cos(x)-\sin(x)}{x^2+\sin^2(x)} dx     \href{https://youtu.be/fZDOBiY57VA}{\,\, => Solution}    \]
 
\[ \footnote{ The standard approach for this integral is to compute recursively} \int_0^{\pi} \frac{1-\cos(nx)}{1-\cos(x)} dx      \href{https://youtu.be/jg108Y1WQDk}{\,\, => Solution}    \]

\[ \footnote{ Integral from 3rd International Mathematics Competition for University Students, 1996} \int_{-\pi}^{\pi} \frac{\sin(nx)}{(1+2^x) \sin(x)} dx , n \ge 0     \href{https://youtu.be/QHTX_IwNPIY}{\,\, => Solution}     \]


\[ \footnote{ An easy modification of famous PUTNAM problem} \int_0^1 \frac{\ln(1+x)}{1+x^2} dx ||  \int_0^{1} \frac{\ln(x)}{1+x^2} dx      \href{https://youtu.be/LbP9ITxI2mQ}{\,\, => Solution}    \]


\[ \footnote{ PUTNAM 1999 A4 series problem} \sum_{m=1}^{\infty} \sum_{n=1}^{\infty} \frac{m^2n}{3^m (n3^m+m3^n)}    \href{https://youtu.be/RiRp4X69McA}{\,\, => Solution}     \]

 
\[ \footnote{ PUTNAM 2016 B6 Double Summation Problem} \sum_{k=1}^{\infty} \frac{(-1)^{k-1}}{k} \sum_{n=0}^{\infty} \frac{1}{k2^n+1}     \href{https://youtu.be/AsJDyy3LU70}{\,\, => Solution}     \]


\[ \footnote{ PUTNAM 1981 B1 A Simple Sum limit } \lim_{n \to \infty} \left[ \frac{1}{n^5} \sum_{h=1}^n \sum_{k=1}^n (5h^4-18h^2k^2+5k^4) \right]    \href{https://youtu.be/rDDichHgk50}{\,\, => Solution}    \]

\[ \footnote{ Use King's Rule and extended King's Rule} \int_0^{\frac{\pi}{2}} \frac{\sin(x)}{\sin(x)+\cos(x)} dx || \int_1^2 \frac{\ln(x)}{x^2-2x+2} dx      \href{https://youtu.be/HDYi4Pr8juo}{\,\, => Solution}     \]

\[ \footnote{ Product of Sines is easily proved using Complex Number} \sin\left( \frac{\pi}{n} \right) \sin\left( \frac{2\pi}{n} \right)......\sin\left( \frac{(n-1)\pi}{n} \right) = \frac{2n}{2^n}      \href{https://youtu.be/dMMNuccYA3E}{\,\, => Solution}     \]

\[ \footnote{ Solving an average IMC inequality} Given:  \int_x^1 f(t) dt \ge \frac{1-x^2}{2} \,\,Prove:  \int_0^1 f^2(t) dt \ge \frac{1}{3}      \href{https://youtu.be/_ScBfvqLLxM}{\,\, => Solution}    \]



\[ \footnote{ A short introduction to Euler Sums Credit: Advanced Integration Techniques by Zaid Alyafeai} S_{p^r,q}= \sum_{k=1}^{\infty} \frac{(H_k^{(p)})^r}{k^q}      \href{https://youtu.be/PzDw_xMZMD4?si=JjC-RSF0Ae1B4HiO}{\,\, => Solution}  \]
 
\[ \footnote{ Generating function associated with the Harmonic number of order p  Credit: Advanced Integration Techniques by Zaid Alyafeai } \sum_{k=1}^{\infty} H_k^{(p)} x^k =  \frac{Li_p(x)}{1-x}   \href{https://youtu.be/JOVsRJtzvQE?si=g68HmY-eEc1QPQa8}{\,\, => Solution}  \]

\[ \footnote{ Using the integral representation of Harmonic number to solve this elegant sum  Credit: Advanced Integration Techniques by Zaid Alyafeai } \sum_{n=1}^{\infty} \frac{H_n}{n^2} = 2 \zeta(3)     \href{https://youtu.be/0MtnAwfWsBM?si=sHZfA4lAZGRcbcdm}{\,\, => Solution}   \]

\[ \footnote{ Can you use the generating function of Harmonic Series to derive this?   Credit: Advanced Integration Techniques by Zaid Alyafeai } \sum_{k=1}^{\infty} \frac{H_k}{k^2} x^k = Li_3(x)-Li_3(1-x)+\log(1-x)Li_2(1-x)+ \frac{1}{2} \log(x)\log^2(1-x)+\zeta(3)     \href{https://youtu.be/yRbdxu2z1to?si=1RKiL3O-TNVh7ex1}{\,\, => Solution}   \]

\[ \footnote{ To be proved, someday in future   Credit: Advanced Integration Techniques by Zaid Alyafeai  } \sum_{n=1}^{\infty} \frac{H_n}{n^q} = \left(1+\frac{q}{2}\right) \zeta(q+1) - \frac{1}{2} \sum_{k=1}^{q-2} \zeta(k+1) \zeta(q-k)     \href{video_link}{\,\, => Solution}   \]

\[ \footnote{ To be proved, someday in future , symmetricity of Euler sums   } S_{p,q}+S_{q,p} = \zeta(p)\zeta(q)+\zeta(p+q) => S_{p,p}=\frac{1}{2}(\zeta^2(p)+\zeta(2p) )    \href{https://youtu.be/oIb1ODrj6xk}{\,\, => Solution}   \]

 
\[ \footnote{ Mellin Transform of ln(x+1)} M_x(ln(x+1))(s) = \frac{\pi cosec(\pi s)}{s}  \href{https://youtu.be/LfpHPMfD6ng?si=Q9YEAX3EiF6IsgOf}{\,\, => Solution}  \]

\[ \footnote{ Mellin Transform of erfc(x) } M_x(erfc(x))(s) = \frac{\Gamma(\frac{s+1}{2})}{\sqrt{\pi}s}    \href{https://youtu.be/eHKGmxPsPHA?si=0MswZRmi6BB05qMT}{\,\, => Solution}  \]

\[ \footnote{ A nice usage of Poly-logarithm} \sum_{n=1}^{\infty} \frac{H_n}{n2^n} = \frac{\zeta(2)}{2}     \href{https://youtu.be/fbi9BsKkq5w?si=SVJlGZ07_uhoDtY0}{\,\, => Solution}   \]
 
 
 \[ \footnote{ Harmonic number and poly-logarithm function might make maniplation easier Credit: Advanced Integration Techniques by Zaid Alyafeai} \int_0^1 \frac{\log^2(1-x) \log(x)}{x} dx = -\frac{\pi^4}{180}     \href{https://youtu.be/_OgTfNdUPAA?si=FVlSH3oYgYZkNuDN}{\,\, => Solution}    \]
 
 
 \[ \footnote{ The ultimate combination of exponential, trigonometric, logarithmic and rational function Credit: Advanced Integration Techniques by Zaid Alyafeai} \int_0^{\infty} e^{-t} \sin(t) \ln(t) \frac{1}{t} dt      \href{https://youtu.be/F4VrfVaQPVY?si=LCe8p5mzlggOiSOg}{\,\, => Solution}   \]
 
\[ \footnote{ Love you if you can see Reimann sum here. Problem from Soviet Union University Student Mathematical Olympiad, 1976} \lim_{n \to \infty} \left( \frac{2^{\frac{1}{n}}}{n+1} +
\frac{2^{\frac{2}{n}}}{n+\frac{1}{2}} + ... + \frac{2^{\frac{n}{n}}}{n+\frac{1}{n}} \right)    \href{https://youtu.be/GXLy2ME-czI?si=5oBqAFQ8sNfdkjz_}{\,\, => Solution}   \]
 
\[ \footnote{ Believe me, this problem will be easier using Riemann sum definition of integration} \int_0^{\pi} \ln(1-2a\cos(x)+a^2) dx    \href{https://youtu.be/KDJFT1AbL9Y?si=wrzif0eUQ8At8jSG}{\,\, => Solution}  \]
 
\[ \footnote{ A tricky inequality from 49th W.L. Putnam Mathematical Competition 2006, proposed by Titu Andresscu }\text{ continuous }f:[0,1] \rightarrow \mathbb{R}.\text{ Find }max \left( \int_0^1 (x^2f(x)-xf^2(x)) dx \right) = \frac{1}{16}    \href{https://youtu.be/7iLxBbG9ZCU?si=WIQuHXqZiGxcTTGa}{\,\, => Solution} \]

 
\[ \footnote{ The true solution of this problem by @mathematician6124 } \int_0^{\frac{1}{2}} \sum_{n=0}^{\infty} ^{n+3}C_n x^n dx      \href{https://youtu.be/lXwUBnvOEjE?si=XczQKnEV7YQDfSYa}{\,\, => Solution}   \]
 
\[ \footnote{ Do you know the integral representation of all these series?} \zeta(s) = \frac{1}{1^s}+\frac{1}{2^s}+\frac{1}{3^s}+\frac{1}{4^s}+\frac{1}{5^s}+.........    \href{video_link}{\,\, => Solution}  \]

\[\eta(s)=  \frac{1}{1^s}-\frac{1}{2^s}+\frac{1}{3^s}-\frac{1}{4^s}+\frac{1}{5^s}+.........   \href{video_link}{\,\, => Solution}  \]


\[\beta(s) =  \frac{1}{1^s}-\frac{1}{3^s}+\frac{1}{5^s}-\frac{1}{7^s}+\frac{1}{9^s}+......... \href{https://youtu.be/XQg8O0Eplu0?si=WFVXj8iS8ZdDgoDJ}{\,\, => Solution} \]



\[(1-\frac{1}{2^s})\zeta(s)= \frac{1}{1^s}+\frac{1}{3^s}+\frac{1}{5^s}+\frac{1}{7^s}+\frac{1}{9^s}+.........   \href{video_link}{\,\, => Solution}  \]

\[ \footnote{ Hope this doesnot scare you. Problem credit :poser.
• 5575: Proposed by Jos´e Luis D´ıaz-Barrero, Barce, school science and math journal } \int_1^{\infty} \frac{dt}{\lfloor t \rfloor^3+9 \lfloor t \rfloor ^2+26 \lfloor t \rfloor +24}     \href{vihttps://youtu.be/tyF9Fmp3xNc?si=XWc_VaZ5DhLRYuiG}{\,\, => Solution}  \]

\[ \footnote{ differentiating geometrically} \frac{d}{dx}(\frac{1}{x})   \href{https://youtu.be/eSAz0RNzMPI?si=fciQQ5VCRFhQn1J_}{\,\, => Solution}  \]

\[ \footnote{ Problem from Stanford Math Tournament 2024} \sum_{m=0}^{\infty} \sum_{n=0}^{\infty} \frac{(\frac{1}{4})^{m+n}}{(2m+1)(m+n+1)}    \href{https://youtu.be/eHN5mXXZ428?si=DvZQZRLg1uAr3PzS}{\,\, => Solution}  \]

\[ \footnote{ This integral will help you get matured} \int_0^{\infty} \frac{\sin(x)}{x+\frac{1}{x}} dx = \frac{\pi}{2e}     \href{https://youtu.be/EK12wPgVWAY?si=AG1FJp4ov0OzQuU6}{\,\, => Solution}  \]

\[ \footnote{ The art of introducing double integrals} \int_0^{\infty} \frac{e^{-x^2}}{(x^2+\frac{1}{2})} dx = \pi \sqrt{\frac{e}{2}} erfc(\frac{1}{\sqrt{2}})  \href{https://youtu.be/enUa_MiYLBU?si=UoYnkalA7RcCbB0U}{\,\, => Solution} \]


 \[ \footnote{ The art of introducing double integrals} \int_0^{\infty} \frac{e^{-x^2}}{(x^2+\frac{1}{2})^2} dx = \sqrt{\pi}    \href{https://youtu.be/enUa_MiYLBU?si=UoYnkalA7RcCbB0U}{\,\, => Solution}  \]
 
 
 \[ \footnote{ The art of introducing double integrals} \int_0^{\infty} \frac{e^{-x^2}}{(x^2+\frac{1}{2})^3} dx = \pi \sqrt{\frac{e}{2}} erfc(\frac{1}{\sqrt{2}}) + \sqrt{\pi}     \href{https://youtu.be/enUa_MiYLBU?si=UoYnkalA7RcCbB0U}{\,\, => Solution}  \]
 
\[ \footnote{ Stanford Maths Tournament 2011} \int_{-\pi}^{\pi} \frac{x^2}{1+\sin(x)+ \sqrt{1+\sin^2(x)} } dx  = \frac{\pi^3}{3}    \href{https://youtu.be/g0zpqV_nNBQ?si=8u1qXhETrYS1095I}{\,\, => Solution}  \]

\[ f(x) = \frac{x^3 e^{x^2}}{1-x^2} \quad f^7(0) = ?  \quad 12600    \href{https://youtu.be/j9laqhBko1c?si=09ZlMO3BYz5Ep61P}{\,\, => Solution}  \]

\[ \footnote{ Everyone can solve the first one. Can you solve the second one?} \int_0^1 \frac{\ln(1+x)}{1+x^2} dx \,\, \int_0^1 \frac{\ln(1+x^2)}{1+x} dx    \href{https://youtu.be/Mm9dypF6zqk?si=AsVAxQrRSnKms2zA}{\,\, => Solution}  \]

\[ \footnote{ Problem proposed by @mathematician6124} \int_0^1 \frac{\ln(1+x) \ln(1+x^2)}{1+x} dx   \href{https://youtu.be/Q2nGmqPxbYo?si=W8mtMTVsp3V0rla2}{\,\, => Solution}   \]
 
\[ \footnote{ Again back with an standard idea} \frac{(2020)^2}{0!}+\frac{(2021)^2}{1!}+\frac{(2022)^2}{2!}+\frac{(2023)^2}{3!}+\frac{(2024)^2}{4!}+......   \href{https://youtu.be/3iw4fbGZZqc?si=Dtxj5-K4vIpM2wOu}{\,\, => Solution}  \]

\[ \footnote{ Let's do one more integral involving poly-logarithm} \int_0^1 \frac{\ln^2(1-x) \ln(x)}{x} dx = -\frac{\pi^4}{180}    \href{https://youtu.be/_OgTfNdUPAA?si=MgYgXdwe1F0d9EEN}{\,\, => Solution}  \]
 
\[ \footnote{ Exotic Integral - I } \int_0^1 x^{x^2} dx   \href{https://youtu.be/9qGkEU-_NXo?si=o_i8XwCZmxhAv3bT}{\,\, => Solution}  \]

 
\[ \footnote{ Exotic Integral - I } \int_0^1 x^{\sqrt{x}} dx    \href{https://youtu.be/eHi7CTjQLN8?si=54A7-Tb8uZgV3uVN}{\,\, => Solution}  \] 
 
 
\[ \footnote{ I am getting s surge of poly-logarithms} \int_0^1 \frac{Li_p(x) Li_q(x)}{x} dx    \href{https://youtu.be/IV4GL0N4BLE?si=lhN7WHQbG2sGfg5z}{\,\, => Solution}  \]
 
\[ \footnote{ Relation between Generalised Harmonic Number and Poly-Gamma function} \sum_{n=1}^{k} \frac{1}{n^p} = H_k^{(p)} = \zeta(p) + (-1)^{p-1} \frac{\psi_{p-1}(k+1)}{(p-1)!}   \href{https://youtu.be/l_FOeVpVtqE?si=oVGhip1Fo6VsStI7}{\,\, => Solution}  \]
 
\[ \footnote{ Integral Representation for r=1} S_{p^r,q} = \sum_{k=1}^{\infty} \frac{(H_k^{(p)})^r}{k^q}   \href{https://youtu.be/m60sp_UNFp8}{\,\, => Solution}  \] 
 
\[ \footnote{ A nice and beautiful symmetric formula}  \sum_{k=1}^{\infty} \frac{H_k^{(p)}}{k^q} + \sum_{k=1}^{\infty} \frac{H_k^{(q)}}{k^p} = \zeta(p) \zeta(q) + \zeta(p+q)   \href{https://youtu.be/oIb1ODrj6xk}{\,\, => Solution}  \]


\[ \footnote{ Finding the value of a Euler sum} \sum_{k=1}^{\infty} \frac{H_k^{(3)}}{k^2} = \frac{11\zeta(5)}{2} - 2 \zeta(2) \zeta(3)   \href{https://youtu.be/3B8NeygXmoA}{\,\, => Solution}  \]

\[ \footnote{ A short and sweet introduction to Sine Integral function} \text{Si}(z)=\int_0^z \frac{\sin(x)}{x} dx \quad \text{si}(z) = - \int_z^{\infty} \frac{\sin(x)}{x}dx \quad \text{Si}(z) = \text{si}(z)+ \frac{\pi}{2}   \href{https://youtu.be/Hon93NHLUAU}{\,\, => Solution} \]

\[ \footnote{ Definition of sinc function} \text{sinc}(x)=
\begin{cases} 
1 & \text{if } x = 0, \\
\frac{\sin(x)}{x} & \text{if } x \neq 0
\end{cases}     \href{https://youtu.be/Hon93NHLUAU}{\,\, => Solution}  
\]

\[ \footnote{ Derivative and antiderivative of sine integral function} \frac{d}{dx}\text{Si}(x) =\text{ sinc}(x)\quad \quad  \int \text{Si}(x) dx = \cos(x) + x \text{Si}(x) + C      \href{https://youtu.be/Hon93NHLUAU}{\,\, => Solution}  
\] 

\[ \footnote{ Doesn't this combination look nice?} \int_0^{\infty}  \sin(x) \text{si}(x) dx = - \frac{\pi}{4}      \href{https://youtu.be/CRZV_IK4MAE}{\,\, => Solution}   \]

\[\footnote{Mellin Transform of si(x)} \int_0^{\infty} x^{\alpha - 1} \text{si}(x) dx = - \frac{\Gamma(\alpha)}{\alpha} \sin(\frac{\pi \alpha}{2})      \href{https://youtu.be/FKH0jXzEDeI}{\,\, => Solution}   \]

\[ \footnote{ Laplace Tranform of si(x)} \int_0^{\infty} e^{-\alpha x } \text{si}(x) dx = - \frac{ \arctan(\alpha)}{\alpha}    \href{https://youtu.be/g2IW5zbp2jc}{\,\, => Solution}
\]


\[ \footnote{ Just one more than euler mascheroni constant} \int_0^{\infty} \text{si}(x) \ln(x) dx = \gamma + 1      \href{https://youtu.be/TEVAYHFuGhs}{\,\, => Solution}
\]

\[ \footnote{ An integration of sin and its sister} \int_0^{\infty} \text{si}(x) \sin(px) dx    \href{https://youtu.be/DwrA8gzjCX0}{\,\, => Solution}  
\]

 
\[ \footnote{ Not the normal sine function} \text{ For a} \neq 1 ,  \int_0^{\infty} \text{si}(x) \cos(ax) dx  = \frac{1}{2a} \ln(\frac{a-1}{a+1})     \href{https://youtu.be/xa8lVsY9rNs}{\,\, => Solution}  \]


 \[ \footnote{ A short and sweet introduction to Cosine Integral Function} \text{ci}(x) =  - \int_x^{\infty} \frac{\cos(t)}{t} dt \quad \text{Cin}(x) = \int_0^x \frac{1-\cos(t)}{t} dt      \href{https://youtu.be/9S5P9UFkvhA}{\,\, => Solution}  \]
 
 \[ \frac{d}{dx}\text{ci}(x) = \frac{\cos(x)}{x} \quad \int \text{ci}(x) dx = x \text{ci}(x)  - \sin(x) + C     \href{https://youtu.be/9S5P9UFkvhA}{\,\, => Solution}   \]
 
 
 \[ \footnote{ You knew the first one. But did you know the second one? }\lim_{ z \to \infty} H_z - \ln(z) = \gamma \quad \quad  \lim_{z \to \infty} \left(  \text{Cin}(z) - \log(z) \right) = \gamma     \href{https://youtu.be/mSIo-frAods}{\,\, => Solution}   \]
 
 \[ \footnote{ One formula that connects them all } \text{Cin}(x) = \gamma + \log(x) - \text{ci}(x)    \href{https://youtu.be/FGVcvunMfZU}{\,\, => Solution}  \]


\[
\footnote{Integrals involving two brothers} \int_0^{\infty} \text{ci}(x) \cos(px) \, dx
 \href{https://youtu.be/rzkje9zbNiw}{\,\, => Solution}   \]

\[
\footnote{Integrals with two rivals} \int_0^{\infty} \text{ci}(px) \text{ci}(x) \, dx \,\,\, \& p>1   \href{https://youtu.be/z31C9Caex80}{\,\, => Solution}  
\]

\[
\footnote{Mellin Transform of $\text{ci}$ function} \int_0^{\infty} x^{\alpha - 1} \text{ci}(x) \, dx = -\frac{\Gamma(\alpha)}{\alpha} \cos\left(\frac{\pi \alpha}{2}\right)   \href{https://youtu.be/HsOTTix-hbo}{\,\, => Solution}  
\]

\[
\footnote{Sweet trick of Feynman} \int_0^{\infty} \text{ci}(x) \log(x) \, dx = \frac{\pi}{2}    \href{https://youtu.be/7zgOssy94wo}{\,\, => Solution}   
\]

\[
\footnote{Laplace Transform of $\text{ci}$ function} \int_0^{\infty} e^{-\alpha x} \text{ci}(x) \, dx = -\frac{1}{2 \alpha} \log(1 + \alpha^2)    \href{https://youtu.be/qXJsH5sslps}{\,\, => Solution} 
\]
 
 \[ \footnote{ Combination of Sine Integral Function and Cosine Integral Function } \int_0^{\infty} \text{si}(qx) \text{ci}(x) dx      \href{https://youtu.be/JvigsMw1tOE}{\,\, => Solution}  \]
 
 \[ \footnote{ The answer is a cool combination of sine and cosine integral function} \int_0^{\infty} \frac{\text{ci}(\alpha x)}{x+\beta} dx = - \frac{1}{2} \{ \text{si}(\alpha \beta)^2 + \text{ci}(\alpha \beta)^2  \}    \href{https://youtu.be/l1fWCF9DQH4}{\,\, => Solution}   \]
 
\[ \footnote{ A short and cozy introduction to Logarithm Integral Function li(x) } \text{li}(x) = \int_0^x \frac{dt}{\log(t)} dt  \href{https://youtu.be/5eSz-RlEB80}{\,\, => Solution}   \] 
 
\[ \footnote { Differential and Integral of Logarithm Integral Function} \frac{d}{dx}\text{li}(z) = \frac{1}{\log(z)} \quad \quad \int \text{li}(z) dz = z \text{li}(z) - \text{Ei}(2 \log(z))    \href{https://youtu.be/5eSz-RlEB80}{\,\, => Solution}   \]
  
\[ \footnote{ A mysteriously simple integral} \int_0^1 \text{li}(z) dz = - \log(2)    \href{https://youtu.be/KKob1Z9f-RE}{\,\, => Solution}   \]

\[  \footnote{ Mellin Transform of Logarithm Integral Function} \int_0^1 x^{p-1} \text{li}(x) dx = - \frac{1}{p} \log(p+1) \href{https://youtu.be/ayGPc5vlTGY}{\,\, => Solution}    \]

\[ \footnote{ Such sums become easier with Complex Summation Technique} \sum_{n=0}^{N-1} cos(n\theta) \quad \quad \sum_{n=1}^{N} cos((2n-1)\theta) \href{https://youtu.be/fLNM_kangwc}{\,\, => Solution}     \]

\[ \footnote{ Problems involving Complex Summation Technique - I} \sum_{n=1}^{N} 2^n \sin(n \theta)  \href{https://youtu.be/Tvd9uqsQ0-w}{\,\, => Solution}    \]

\[ \footnote{ Problems involving Complex Summation Technique - II} \sum_{n=0}^{\infty} 4^{-n} \cos(\frac{n \pi}{3})   \href{https://youtu.be/EyxH-sgL-fI}{\,\, => Solution}   \]

\[ \footnote{ Problems involving Complex Summation Technique - III} \sum_{n=0}^{\infty} 2^{-n} \sin(\frac{n \pi}{3}) \href{https://youtu.be/xOJL3T7b52k}{\,\, => Solution}    \]

\[ \footnote{ Problems involving Complex Summation Technique - IV} \sum_{n=0}^{\infty} 2^{-n} \sin(\frac{n \pi}{2})    \href{https://youtu.be/ZcPIDpM7MN0}{\,\, => Solution}   \]

\[ \footnote{ The general trick to deal with Logarithm Integral Function} \int_0^1 \text{li}(\frac{1}{x}) \sin(a\log(x)) dx     \href{https://youtu.be/T0yvL3-huq0}{\,\, => Solution}   \]

\[ \footnote{ Will Feynman be useful in this integral} \int_0^1 \frac{\text{li}(x)}{x} \log^{p-1}(\frac{1}{x}) dx      \href{https://youtu.be/HUzoanKWuJU}{\,\, => Solution}    \]

\[ \footnote{ Combination of Logarithm and Logarithm integral function} \int_1^{\infty} \text{li}(\frac{1}{x}) \log^{p-1}(x) dx \href{https://youtu.be/Uz_muLZ5NWo}{\,\, => Solution}   \]


\[ \footnote{ Same same but different} \int_0^1 \text{li}(x) \log(x) dx = \log(2) - \frac{1}{2}    \href{https://youtu.be/EyqSAiBjugk?si=cJ4EvtdEkAjyUKNc}{\,\, => Solution}    \]

 
\[ \footnote{ A short introduction to Clausen Function}  Cl_m(\theta)=
\begin{cases} 
\sum_{k=1}^{\infty} \frac{\sin(k \theta)}{k^m} & \text{if  m is even}, \\
\sum_{k=1}^{\infty}\frac{\cos(k \theta)}{k^m} & \text{if m is odd}
\end{cases} 
\href{https://youtu.be/CArVh0JszsU?si=WZ85vgTYTxbBhx7O}{\,\, => Solution}  \]
  

\[   Sl_m(\theta)=
\begin{cases} 
\sum_{k=1}^{\infty} \frac{\cos(k \theta)}{k^m} & \text{if  m is even}, \\
\sum_{k=1}^{\infty}\frac{\sin(k \theta)}{k^m} & \text{if m is odd}
\end{cases}
 \href{https://youtu.be/CArVh0JszsU?si=WZ85vgTYTxbBhx7O}{\,\, => Solution}
\]

\[ Li_m(e^{i \theta}) =
 \begin{cases}
 Sl_m(\theta) + i Cl_m(\theta) & \text{ if m is even}, \\
Cl_m(\theta) + i Sl_m(\theta) & \text{ if m is odd} 
\href{https://youtu.be/LZ9yN17Qp24?si=ORfefqMeS642yWKO}{\,\, => Solution}

\]

\[ \frac{d}{d\theta} ( Cl_2(\theta) ) = - \log(2 \sin(\frac{\theta}{2}) ) \quad \quad Cl_2(\theta) = - \int_0^{\theta} \ln|2 \sin(\frac{x}{2})| dx   \href{video link here}{\,\, => Solution} \]
\[ Cl_2(\theta+ 2m \pi) = Cl_2(\theta) || Cl_2(-\theta) = - Cl_2(\theta) \href{https://youtu.be/IjpskJTzkkw?si=AGxGr30A2-tHpfJh}{\,\, => Solution} \]


\[ \footnote{ Reflection formula of Clausen Functions} Cl_m(2 \theta) = 2^{m-1} ( Cl_m(\theta) - (-1)^m Cl_m(\pi - \theta) ) || Cl_2(2\theta) = 2 ( Cl_2(\theta) - Cl_2(\pi - \theta) ) \href{https://youtu.be/0JEqATfx8vk?si=kOJU1EbTK6YAJEFI}{\,\, => Solution} \]

\[ \footnote{ Integral of Clausen Function} \int_0^{\pi} Cl_m(\theta) d \theta } \hrefhttps://youtu.be/lUFB3NDjCiM?si=weYX_giY8QmwG2C4}{\,\, => Solution} \]

\[ \footnote{ Laplace Transform of Clausen Function} \text{ If m is even, find: } \int_0^{\infty} Cl_m(\theta) e^{-n \theta} d\theta \href{https://youtu.be/t6HmKkDwICI?si=WXFU7rKmUVVEHkWB}{\,\, => Solution} \]

\[ \footnote{ Introduction to Clausen Integral Function} Cl_2(x) = \sum_{k=1}^{\infty} \frac{\sin(kx)}{k^2} \quad || Cl_2(\theta)= - \int_0^{\theta} \log(2 \sin(\frac{x}{2})) dx \quad || Cl_2(\frac{\pi}{2}) = G \href{https://youtu.be/1FjopAXXnmg?si=IpRQYUpnC4REZvug}{\,\, => Solution} \]


\[ \footnote{ Two different derivations for this identity} Cl_2(\theta)= \sum_{k=1}^{\infty} \frac{\sin(k \theta)}{k^2} = - \int_0^{\theta} \log(2 \sin(\frac{x}{2})) dx \href{https://youtu.be/HE5b3faaVGk?si=kU6UC-lc9vG691hC}{\,\, => Solution} \]

\[ \footnote{ Derivation of reflection formula of Clausen Integral Function using Sum and Integral Definition} Cl_2(2 \theta) = 2 Cl_2(\theta) - 2 Cl_2(\pi - \theta) \href{https://youtu.be/8IkY46X9IHM?si=Mj1UL8qpJ2IuQ5c-}{\,\, => Solution} \]

\[ \footnote{ Square of Clausen Integral Function} \int_0^{2 \pi}  Cl_2(x) ^2 dx = \frac{\pi^5}{90} \href{https://youtu.be/Xzrk_nf8ugQ?si=ZQ7kIAOLqRLzel86}{\,\, => Solution} \]

\[ \footnote{ This is when Clausen Integral Function can be useful} \int_0^{\frac{\pi}{2}} x \log(  \sin(x) ) dx = \frac{7}{16} \zeta(3) - \frac{\pi^2}{8} \log(2) \href{https://youtu.be/6dgQgPajY4I?si=HmQvIKRNCXZHr4dA}{\,\, => Solution} \]

 \[ \footnote{ This is the best match to use Clausen Integral Function} \int_0^{\frac{\pi}{4}} x \cot(x) dx = \frac{1}{8} ( \pi \ln(2) + 4 G ) \href{https://youtu.be/CJ1jAC-7upw?si=_x9N37d5qfCTEJ_h}{\,\, => Solution} \]
 
  
\[ \footnote{ Finding all the beta values } \beta(1), \beta(2), \beta(3), \beta(4), \beta(5), \beta(6), \beta(7), \beta(8), \beta(9), \beta(10), \beta(11), \beta(12), \beta(13), \beta(14), \beta(15),  ... \href{https://youtu.be/C_Ia_IebLHk?si=7IeIRgYy-X2_jQpB}{\,\, => Solution} \]

\[ \footnote{ Sum and Integral representation of beta function} \beta(s) = \sum_{n=0}^{\infty} \frac{(-1)^n}{(2n+1)^s} \href{video link here}{\,\, => Solution} \]
\[ \beta(s) = \frac{1}{\Gamma(s)} \int_0^{\infty} \frac{x^{s-1}}{e^{-x} + e^{x}} dx  \href{video link here}{\,\, => Solution} \]
\[ \beta(s) = \frac{1}{\Gamma(s)} \int_0^1 \frac{(-\ln(x))^{s-1}}{1+x^2} dx  \href{https://youtu.be/yj7ZUHRiAJ0?si=R6ilrsc3Oiju5GOz}{\,\, => Solution} \]

\[ \footnote{ Representation of beta function in terms of Hurwitz zeta function}\zeta(s,\alpha) = \sum_{n=0}^{\infty} \frac{1}{(n+\alpha)^s} \quad \beta(s) = \frac{1}{4^s} [ \zeta(s,\frac{1}{4}) - \zeta( s,\frac{3}{4}) ] \href{https://youtu.be/bPcKi-fRQLE?si=n8Iz922X8Rnb1d5s}{\,\, => Solution} \]

\[ \footnote{ Representation of beta function in terms of Lerch transcendent function} \Phi(z,s,\alpha) = \sum_{n=0}^{\infty} \frac{z^n}{(n+\alpha)^s} \quad \beta(s) = 2^{-s} \Phi(-1,s,\frac{1}{2})  \href{https://youtu.be/Ojp9P6ary5Y?si=nv1GGV4gtvkHf2YT}{\,\, => Solution} \]

\[ \footnote{ Representation of beta function in terms of Poly-Logarithm Function} Li_p(z) =\sum_{n=1}^{\infty}  \frac{z^n}{n^p} \quad \beta(s) = \frac{i}{2} \left( Li_s(-i) - Li_s(i) \right) \href{https://youtu.be/w7tQj4rkXGM?si=CMnnwMqpZOahe9RR}{\,\, => Solution} \]

\[ \footnote{ Representation of beta function in terms of Poly-gamma function } \sum_{n=0}^{\infty} \frac{1}{(n + \alpha)^s} = \frac{(-1)^s \psi^{s-1} ( \alpha) }{(s-1)!} \quad \beta(s) = \frac{1}{(-4)^s (s-1)!} [ \psi^{s-1} ( \frac{1}{4}) - \psi^{s-1} ( \frac{3}{4})] \href{https://youtu.be/WrGP94zjHSE?si=qJxRSi4nX1drnb7d}{\,\, => Solution} \]

\[ \footnote{ A cool use of King's Rule} \int_0^1 cot^{-1}(1-x+x^2) dx = \frac{\pi}{2} - \log(2) \href{https://youtu.be/t0QXd5mxXt4?si=yXcnpa5MkisuQDI6}{\,\, => Solution} \]

\[ \footnote{ Everyone can do the first integral. Can you do the second one?  Solution Credit: @SussySusan-lf6fk } \int_0^{\pi} \frac{x \sin(x)}{1+\cos^2(x)} dx \,\, \int_0^{\frac{\pi}{2}} \frac{x \sin(x)}{1+\cos^2(x)} dx \href{https://youtu.be/wV54mOMMiGM?si=vpQulcTBGkden77X}{\,\, => Solution} \]

\[ \footnote{ Strange but useful integral representation of Clausen Integral Function} Cl_2{\theta} = - \sin(\theta) \int_0^1 \frac{\log(x)}{x^2-2\cos(\theta) x +1 } dx \href{https://youtu.be/m_4dzTlo4jE?si=c33xTauEWCyH7K2q}{\,\, => Solution} \]

\[ \footnote{ Some special values of Cl_2(\theta)} Cl_2(0) = 0 || Cl_2(\pi) = 0 || Cl_2(\frac{\pi}{2}) = G || Cl_2(\frac{-\pi}{2}) = -G  || Cl_2(\frac{3\pi}{2}) = -G  \href{https://youtu.be/IAPNYHwJCkc?si=QHwSD9sdveiCbebW}{\,\, => Solution} \]

\[ Cl_2(\frac{2\pi}{3}) = - \frac{1}{6\sqrt{3}} \left( \psi'(\frac{2}{3}) - \psi'(\frac{1}{3}) \right) \href{https://youtu.be/IAPNYHwJCkc?si=QHwSD9sdveiCbebW}{\,\, => Solution} \]

\[ Cl_2(\frac{\pi}{3}) = -\frac{1}{24\sqrt{3}} [ - \psi'(\frac{1}{6}) - \psi'(\frac{1}{3}) + \psi'(\frac{2}{3}) + \psi'(\frac{5}{6}) \href{https://youtu.be/IAPNYHwJCkc?si=QHwSD9sdveiCbebW}{\,\, => Solution} \]

\[ \footnote{ Doing this Feynman-Hibbs Integral will make you an quantum theory expert} \int_0^{T} \frac{e^{\frac{-a}{T-\tau} - \frac{b}{\tau}}}{(T-\tau)^{\frac{1}{2}} \tau^{\frac{3}{2}}} d\tau = \sqrt{\frac{\pi}{bT}} e^{-\frac{1}{T} (\sqrt{a} + \sqrt{b})^2 \href{https://youtu.be/lvmVau0w7qE?si=5xHA7BYxafwkqmZI}{\,\, => Solution} \]


\[ \footnote{ A single integral will solve all these integrals} \int_0^{\infty} \sin(t^2) dt = \int_0^{\infty} \cos(t^2) dt = \frac{\sqrt{\pi}}{2\sqrt{2}}  \href{https://youtu.be/VkTC_8y2eBM?si=lf7rtD4rYaEFoZ-Q}{\,\, => Solution} \]

\[ \int_0^{\infty} \sin(t^2 - \frac{1}{t^2}) dt = \int_0^{\infty} \cos(t^2 - \frac{1}{t^2}) dt = \frac{\sqrt{\pi}}{2\sqrt{2} e^2} \href{https://youtu.be/VkTC_8y2eBM?si=lf7rtD4rYaEFoZ-Q}{\,\, => Solution} \]

\[ \int_0^{\infty} \sin(t^2 + \frac{1}{t^2}) dt = \frac{\sqrt{\pi}}{2} \sin(\frac{\pi}{4}+2) || \int_0^{\infty} \cos(t^2 + \frac{1}{t^2}) dt = \frac{\sqrt{\pi}}{2} \cos(\frac{\pi}{4}+2)   \href{https://youtu.be/VkTC_8y2eBM?si=lf7rtD4rYaEFoZ-Q}{\,\, => Solution}  \]

\[ \int_0^{\infty} e^{-pt^2 - \frac{q}{t^2}} dt = \frac{1}{2} \sqrt{\frac{\pi}{p}} e^{-2\sqrt{pq}}  \href{https://youtu.be/VkTC_8y2eBM?si=lf7rtD4rYaEFoZ-Q}{\,\, => Solution} \]
 
\[ \footnote{ MAZ identity is back} \int_0^{\infty} \frac{2 e^{-x^2 \sqrt{3}}\sin(3x^2)}{x} dx = \frac{\pi}{3}  \href{https://youtu.be/7PX5hiskO9A?si=EdQU-i6J2SPmYzQh}{\,\, => Solution}  \]


\[ \footnote{ An integral with answer as Lemniscate constant} \int_0^{\infty} \frac{ e^{-x} tanh(x)}{x} dx = \ln ( \frac{\varpi^2}{\pi})  \href{https://youtu.be/3fRBzogRZzA?si=pQ0hyy21aZWXXBVc}{\,\, => Solution} \]

\[ \footnote{ There is a nice little idea involved in this integral} \int_0^{\frac{1}{2}} \frac{\ln(1+x) \ln(x)}{x} dx = \ln(2) Li_2(-\frac{1}{2}) + Li_3(-\frac{1}{2})  \href{https://youtu.be/kPh0KjicGNE?si=IFtvw_7L9BgK1dnH}{\,\, => Solution} \]

\[  \footnote { A problem from the PREFACE section of In Pursuit of Zeta 3 } \int_0^1 x^{5} \ln(1+x) dx = \frac{74}{720} \href{https://youtu.be/THT6W94ZvN0?si=qio-EiQRueWmZrQQ}{\,\, => Solution} \]

\[ \footnote{ An interesting way to solve these system of equations} \frac{\sqrt{3}}{2} x - \frac{1}{2}y = 1 \]
\[ \frac{1}{2}x + \frac{\sqrt{3}}{2} y = 2  \href{video link here}{\,\, => Solution} \]

\[ \footnote{ Proving the results stated by Jacob Bernoulli } \sum_{k=1}^{\infty} \frac{k}{2^k} = 2 || \sum_{k=1}^{\infty} \frac{k^2}{2^k} = 6  || \sum_{k=1}^{\infty} \frac{k^3}{2^k} = 26     || \sum_{k=1}^{\infty} \frac{k^4}{2^k} = 150  \href{https://youtu.be/bfifVSke_S4?si=fSolNBRBMG8UoKBd}{\,\, => Solution} \]

\[ \footnote{ Problem from the School science and Math Journal : Problem proposed by Ovidiu Furdui and Alina Sînt˘ am˘arian} \text{ Calculate }S = \sum_{n=1}^{\infty} (2n-1) \left( \sum_{k=0}^{\infty} \frac{(-1)^k}{(n+k)^2} \right) ^2 = \frac{\pi^2}{12}  \href{https://youtu.be/XnZqANMnnUU?si=Afh9OQV20N48TQGs}{\,\, => Solution} \]

\[ \footnote{ I bet you know this .... But what about this?}  \frac{1}{1}+\frac{1}{2}+\frac{1}{3}+\frac{1}{4}+\frac{1}{5}+\frac{1}{6}+..... = Diverges  \href{https://youtu.be/QUOqoqSv00U?si=C29eqf2PizX6rgy-}{\,\, => Solution} \]

\[ \frac{1}{2} + \frac{1}{3} + \frac{1}{5} + \frac{1}{7} + \frac{1}{11} + \frac{1}{13} + \frac{1}{17} + \frac{1}{19} + ...... = ? \href{https://youtu.be/QUOqoqSv00U?si=C29eqf2PizX6rgy-}{\,\, => Solution}  \]

 \[\footnote{ Can we make sense of this ?} \sum_{n=0}^{\infty} \binom{2n}{n} = \frac{-1}{\sqrt{3}} i  \href{https://youtu.be/S_ftj2ws1V8?si=vglKK1ycBhlVv7BF}{\,\, => Solution} \]
\[ \sum_{n=0}^{\infty} (-1)^n \binom{2n}{n} = \frac{1}{\sqrt{5}} \href{https://youtu.be/S_ftj2ws1V8?si=vglKK1ycBhlVv7BF}{\,\, => Solution} \]

\[ \footnote{ I love how the simple tool of differentiation can become powerful at times} \sum_{n=1}^{\infty} \frac{(n+1)}{(-4)^{n+1}} \zeta(n+1) = \frac{G}{2} + \frac{\pi^2}{16} - \frac{\pi}{8} - \frac{3}{4} \ln(2) \href{https://youtu.be/p77jYmAW8XU?si=SVDpIdeBm4zYJSIQ}{\,\, => Solution} \]

\[ \footnote{ Finding these frequently used Digamma Values} 
\psi(1)
\psi(\frac{1}{2}) \psi(\frac{1}{3}) 
 \psi(\frac{2}{3}) 
 \psi(\frac{1}{4}) \psi(\frac{3}{4}) 
 \psi(\frac{1}{6}) 
 \psi(\frac{5}{6}) \href{https://youtu.be/s86vAS54qAM?si=Lq9q46974iWGww1l}{\,\, => Solution}  \]


\[ \footnote{ Using Matrix Multiplication to find the nth fibonacci Number} This can be used to prove that fact that F_{(n+1)} F_{(n-1)} - F_{n}^2 = (-1)^n \href{https://youtu.be/MxfBWrcC8Jw?si=DmZnZUAZzrnx-Fcg}{\,\, => Solution} \] 

\[ \footnote{ Some more oneliner proofs}  If L_n be Lucas number with L_n = L_{n-1} + L_{n-2}; L_0 = 2, L_1 = 1\]
\[ show that L_{n+1}L_{n-1} - L_{n}^2 = 5(-1)^{n-1}. \href{https://youtu.be/Gab6K1p_ymo?si=jCqmqG30NIvW_NZk}{\,\, => Solution} \]

\[ If P_n be pell number with P_n = 2 P_{n-1} + P_{n-2} with P_0 = 0 and P_1 = 1\]
\[ show that P_{n+1}P_{n-1} - P_{n}^2 = (-1)^n.  \href{https://youtu.be/Gab6K1p_ymo?si=jCqmqG30NIvW_NZk}{\,\, => Solution} \]

\[ \footnote{ one liner proof of the extended Fibonacci type theorem} if F_n = k . F_{n-1} + F_{n-2}; F_0 = 0, F_1 = 1\]
\[ then prove that  F_{n+1} . F_{n-1} - F_{n}^2 = (-1)^n  \href{https://youtu.be/Gab6K1p_ymo?si=jCqmqG30NIvW_NZk}{\,\, => Solution} \]

\[ \footnote{ Proving this with Matrix // I can't believe how satisfied I am right now} If F_n be the nth fibonacci number defined by F_n = F_{n-1} + F_{n-1}; F_0 = 0 , F_1 = 1, then \href{https://youtu.be/QKf_91q0wWw?si=hBLT2CgccWtq7JHe}{\,\, => Solution} \]

 \[ prove that : \sum_{i=0}^{n} F_i = F_{n+2} - 1  \href{https://youtu.be/QKf_91q0wWw?si=hBLT2CgccWtq7JHe}{\,\, => Solution} \]

\[ \footnote{ I am just loving this stuff: proving fibonacci theorems using Matrix Multiplication} 
\sum_{i=0}^{n} F_{2i+1} = F_{2n+2}  \& \sum_{i=0}^{n} F_{2i} = F_{2n+1}-1
\href{https://youtu.be/gsnRYTmtHiE?si=lnmSVDAexZ4gxd-9}{\,\, => Solution} \]

\[ \footnote{d’Ocagne’s Identity:} 
F_{m+n} = F_{m+1}F_{n}+F_mF_{n-1}
\href{https://youtu.be/FpBne2kwzbo?si=twUPslqmdJtPhOkY}{\,\, => Solution} 
\]


\[ \footnote{Sum of fibonaccci numbers with weighted index} \sum_{i=1}^{n} i.F_i = n F_{n+2} - F_{n+3} + 2
\href{https://youtu.be/0IGBllokQso?si=DPSK2WYFDPP8Ewbq}{\,\, => Solution} \]

\[ \footnote{ Don't get tricked to using Trig - sub} \int_{\frac{\pi}{4}}^{\frac{\pi}{2}} \frac{dx}{\sin^2(x) (\sin^2(x)+1)(\sin^2(x)+2)}  \href{https://youtu.be/WLP0NGelc5g?si=dgPUWAGXDNa9sgk0}{\,\, => Solution} \]

\[ \footnote{ You know, Trigs are really helpful sometimes} \int_0^{\infty} \frac{dx}{ ( x + \sqrt{1+x^2})^n} , n>1
\href{https://youtu.be/B69emOxHcJY?si=jRM0Zs8PKaavmItL}{\,\, => Solution}  \]

\[ \footnote{ My friend's Homework Question} \int_1^{\infty} \frac{dx}{(1+x)\sqrt{1+2ax+x^2}}, a > 1    \href{https://youtu.be/lmDXj5OxqpY?si=DB-HWL1EfiAOiCn5}{\,\, => Solution} \]

\[ \footnote{ let's prove this interesting result:} 0 + 1 + 1 + 2 + 3 + 5 + 8 + 13 +\]
\[ 21 + 34 + 55 + 89 + 144 +....+F_n+.. = -1
\href{https://youtu.be/ET9lVR8AWBY?si=z_jDgjvX-Jhpw-GI}{\,\, => Solution} \]

\[ \footnote{ This result is satisfying to prove} \sum_{k=1}^{n} (-1)^k k^2 = (-1)^n \frac{n(n+1)}{2}
\href{https://youtu.be/oITRxPhIo10?si=r_C0PujiqMxLh37M}{\,\, => Solution} \]

\[ \footnote{ Everyone knows about Bernoulli numbers but do you know the Euler numbers ?}
\href{video link here}{\,\, => Solution} \]


\[ \footnote{ Three ways to find Bernoulli Numbers} 
1^4 + 2^4 + 3^4 + .... + (n-1)^4 = \frac{1}{5} \left[ \binom{5}{0}B_0 n^5 + \binom{5}{1} B_1 n^4 + \binom{5}{2} B_2 n^3 + \binom{5}{3} B_3 n^2 + \binom{5}{4}B_4 n        \right] \href{https://youtu.be/5zhoOp_Afo8?si=T7260asZ67BW6LHP}{\,\, => Solution} 
\]


\[ \footnote{ How people prove this in late transcendentals method without using exponential function? } \ln(ab) = \ln(a) + \ln(b)  \href{https://youtu.be/fGfRB6QsTN4?si=UaTkgUNmG2jCUBxE}{\,\, => Solution} \]

\[ \footnote{ Inventing math to prove the irrationality of the root(2)}  \sqrt{2}
\href{https://youtu.be/t-jGRKXRTIU?si=hHqSdygkZykx9CcK}{\,\, => Solution} \]


\[ \footnote{ A Modern Solution to Basel Problem} https://www.youtube.com/watch?v=5-pXwWNcsbc
\href{video link here}{\,\, => Solution} \]

\[ \footnote{ Finding these partial derivatives in 3 seconds, I am not lying} 3x^2yz+3e^xz+3\ln(y)z = 0 , \frac{\partial z}{\partial x} = ?  \frac{\partial z}{\partial y} = ?   \href{https://youtu.be/VEDwvQoOzHk?si=sONV0odKEGGTqWBp}{\,\, => Solution} \]

\[ \footnote{ I taught this in my class today} \sum_{n=1}^{\infty} \frac{1}{n^2} = \int_0^1 \int_0^1 \frac{1}{1-xy} dx dy =  \frac{\pi^2}{6}  \href{https://youtu.be/rFP9y1kBC1I?si=742NsRDzgoZFVZFn}{\,\, => Solution} \]

\[ \footnote{ Complex Numbers are the best friends of mankind!!} \frac{d^n}{dx^n}(e^x \sin x)   \href{https://youtu.be/FpZ29orHNBM?si=OKTawutb-0uJzVQN}{\,\, => Solution} \]

\[ \footnote{ Finding formula for surface area and volume of sphere using Calculus} 1. Volume (Disc Method, Shell Method, Icecream Method) \]
\[ 2. Surface Area (Ring Method, Surface Area element method) 
\href{https://youtu.be/K31ZeIwlztM?si=1ui7A0KXaKawufAw}{\,\, => Solution} \]

\[ \footnote{ Using calculus to find volume and surface area of Dough nut (Torus)} 1. Surface area element method and volume element method   \href{https://youtu.be/cpz3rB7PzmI?si=zbTX723rHSTAl-wO}{\,\, => Solution} \]

\[ \footnote{ Chill Guys chill, Jacobian is just a scalar Triple product}
}
dV = \left[ \left(\frac{\partial x}{\partial u} du , \frac{\partial y}{\partial u} du ,\frac{\partial z}{\partial u} du \right) \times \left( \frac{\partial x}{\partial v} dv, \frac{\partial y}{\partial v} dv, \frac{\partial z}{\partial v} dv \right) \right] . \left( \frac{\partial x}{\partial w} dw, \frac{\partial y}{ \partial w } dw, \frac{\partial z}{ \partial w} dw \right) 
  \href{https://youtu.be/DFXJ-rME0bg?si=hZF-Y8nKv4B1LFb8}{\,\, => Solution}  \]

\[ \footnote{ n-D volume of n-D parallelopiped spanned by n- linearly independent vectors} Finding 4-D volume of 4-D parallelopiped spanned by  \, \vec{a}, \vec{b}, \vec{c}, \vec{d} \]
\[ <1,1,2,3>,<1,2,3,5>,<2,3,5,8>,<3,5,8,13>  \href{video link here}{\,\, => Solution}  \href{https://youtu.be/sBTU3UJhILM?si=qhCmGXQuoDeUOXtL}{\,\, => Solution}\]

\[ \footnote{ Complex Numbers are best friends of humanity - II}  \int_0^{\infty} x^m e^{-x} \sin(x) dx = \frac{m!}{\sqrt{2^{m+1}}} \sin\frac{(m+1)\pi}{4}\href{https://youtu.be/6aZMsa9W9hA?si=Le03ockvRksSuclS}{\,\, => Solution}
\]

\[ \footnote{ Complex Numbers are best friends of Humanity - III } \int_0^{\pi} \frac{1-a\cos\theta}{1-2a\cos\theta+a^2} d\theta\href{https://youtu.be/ddLhdinA0wI?si=Xaai1GlXarPzoB4h}{\,\, => Solution}
\]

\[ \footnote{ Love you to infinity!!!  Intense Satisfaction}\text{ If} f(x,y,z) = 0 \, \& \, g(x,y,z)=0 \]
\[ \frac{dy}{dx} =\frac{ \frac{\partial g}{\partial x} \frac{\partial f}{\partial z} - \frac{\partial f}{\partial x} \frac{\partial g}{\partial z} } { \frac{\partial f}{\partial y} \frac{\partial g}{\partial z} - \frac{\partial g}{\partial y} \frac{\partial f}{\partial z}  }  \href{https://youtu.be/g9QvbkVXPLg?si=MZZpV-WP_04RGqQi}{\,\, => Solution}
\]


\[ \footnote{Partial differentiation comes to rescue} \int_0^{\frac{\pi}{2}} \frac{d\theta}{(x^2\cos^2\theta + y^2 \sin^2\theta)^2} = \frac{\pi}{4xy} \left( \frac{1}{x^2} + \frac{1}{y^2} \right) \href{https://youtu.be/Og4c7dVIryA?si=GVwU29pE1mM_a2qt}{\,\, => Solution} \]


\[ \footnote{ Find these partials in two different ways} x = r cosh\theta , y = r sinh\theta , \frac{\partial r}{\partial x} , \frac{\partial r}{\partial y} , \frac{\partial \theta}{\partial x} , \frac{\partial \theta}{\partial y} \href{https://youtu.be/4RjS8n5ZgDg?si=HE2SeB7R5AhHHnua}{\,\, => Solution}\]

\[ \footnote{ WTF ??? } x = r cosh\theta , y = r sinh\theta , V = V(x,y) \]
\[ \frac{\partial^2V}{\partial x^2} - \frac{\partial^2V}{\partial y^2} = \frac{\partial^2V}{\partial r^2} + \frac{1}{r} \frac{\partial V}{\partial r} - \frac{1}{r^2} \frac{\partial^2 V}{\partial \theta^2} \href{https://youtu.be/_xj_NzjQqfU?si=e4yeqoDOrjjucZwt}{\,\, => Solution} \]

\[ \footnote{ Trick to find reciprocated partials} x = r \sin\theta \cos\phi, y = r \sin\theta \sin\phi , z = r \cos\theta; \frac{\partial r}{\partial x} , \frac{\partial \theta}{\partial x}, \frac{\partial \phi}{\partial x}, \frac{\partial r}{\partial y}, \frac{\partial \theta}{\partial y}, \frac{\partial \phi}{\partial y} \href{https://youtu.be/KKdPCc31Ji4?si=iHnMkKIjS344kSI3}{\,\, => Solution}
\]

\[ \footnote{ Gauss Jordan Method / Easiest Method to calculate Matrix Inverse} [A\,|\,I] \href{https://youtu.be/uyqVRJ7rAoY?si=RWLfpmJUNgd_9d-s}{\,\, => Solution}
\]

\[ \footnote{ Our Professor gave this as classwork, WTF}  z = \frac{x^{\frac{1}{4}}+y^{\frac{1}{4}}}{x^{\frac{1}{5}}+y^{\frac{1}{5}}}. Find: x \frac{\partial z}{\partial x} + y \frac{\partial z}{\partial y} = ? \href{https://youtu.be/a11BcfHVNM4?si=dZh_eo3eWvzEFjNA}{\,\, => Solution}
\]

\[ \footnote{ This will be one of your unforgettable maths journey} \cos \begin{pmatrix} 7&3 \\ 3 & -1  \end{pmatrix}   \href{https://youtu.be/2dodh3ErQQk?si=9FE7DwVmiZvGmoYM}{\,\, => Solution}
\]

\[ \footnote{ Five different ways to compute determinant} 
\det\begin{pmatrix}
1 & 2 & 3 \\
1 & 1 & 3 \\
2 & 1 & 2
\end{pmatrix} \href{https://youtu.be/v6EE9fdeya4?si=DyYAr_NGvNDUmjEZ}{\,\, => Solution}
\]

\[ \footnote{ In search of gold, we lost diamond} (uv)''' = \binom{3}{0} u'''v + \binom{3}{1} u''v' + \binom{3}{2} u'v'' + \binom{3}{3}uv''' \href{https://youtu.be/xJ6rWfdLRZ0?si=Cy8UaZ5Lh2_46L9m}{\,\, => Solution}
\]

\[ (\frac{u}{v})''' = - \frac{1}{v^4}
\det \begin{vmatrix}
    u & v & 0 & 0 \\
    u' & v' & v & 0 \\
    u'' & v'' & 2v' & v \\
    u''' & v''' & 3v'' & 3v'
\end{vmatrix}  \href{https://youtu.be/xJ6rWfdLRZ0?si=Cy8UaZ5Lh2_46L9m}{\,\, => Solution}

\]

\[ \footnote{ Just in case if you are still wondering why this was true in the first place } \vec{a} = <a_1,a_2,a_3> , \vec{b} = <b_1,b_2,b_3>   \]

\[ a_1b_1+a_2b_2+a_3b_3 = ||\vec{a}|| \,||\vec{b}|| \cos\theta   \href{https://youtu.be/mKTgnHs256c?si=HN0-nefXFuwkWtoP}{\,\, => Solution} 
 \]

\[ \left| \begin{vmatrix}
\hat{i} & \hat{j} & \hat{k} \\
a_1 & a_2 & a_3 \\
b_1 & b_2 & b_3
\end{vmatrix} \right| = ||\vec{a}||\, ||\vec{b}|| \sin\theta   \href{https://youtu.be/mKTgnHs256c?si=HN0-nefXFuwkWtoP}{\,\, => Solution}
\]

\[ \footnote{ do it without doing}
\det\begin{vmatrix}
(y+z)^2 & x^2 & x^2 \\
y^2 & (z+x)^2 & y^2 \\
z^2 & z^2 & (x+y)^2
\end{vmatrix}   \href{https://youtu.be/VF60KIVw3k4?si=SyNBWNiKA1d3qq_a}{\,\, => Solution}
\]


\[ \footnote{ Answer won't have n !! } 
\begin{vmatrix}
1 & a & a^2 \\
\cos((n-1)x) & \cos(nx) & \cos((n+1)x) \\
\sin((n-1)x) & \sin(nx) & \sin((n+1)x)
\end{vmatrix}   \href{https://youtu.be/NTI4vrqSXAI?si=kf9n2zoI1S8Vzg9N}{\,\, => Solution}
\]

\[ \footnote{ Solving these system of differential equations in two different ways
i) Exponential Ansatz Method
ii) Using Matrix } \frac{dx}{dt} = 5x + y
\]
\[ \frac{dy}{dt} = x + 5y \href{https://youtu.be/RMjdKQT8tVE?si=USPl_VPws1uz8k0e}{\,\, => Solution}\]

\[ \footnote{ Find non trivial solutions for x, y and z }
\begin{cases}
\lambda x + y + \sqrt{2} z = 0, \\
x + \lambda y + \sqrt{2} z = 0, \\
\sqrt{2} x + \sqrt{2} y + (\lambda - 2) z = 0.
\end{cases}  \href{https://youtu.be/2k7cHVcst3Q?si=T4QigJSU7-j23yzY}{\,\, => Solution}
\]

\[ \footnote{ Third is slightly different!} y''-2y'-8y = 0 \]
\[ y''-2y'-8y = e^{-x} \]
\[ y''-2y'-8y=e^{-2x}   \href{https://youtu.be/8qT-_tRRuH4?si=TiV4xjd7DFrPuUtW}{\,\, => Solution}
\]

\[ \footnote{ University of Ibadan, ODE Integration Bee, Finals, problem 9} a>0, b>0  \int_0^{\infty} \frac{\log(1+a^2x^2)}{1+b^2x^2} dx   \href{https://youtu.be/yIGxzvRZj0A?si=F8t-5_C-aQJtMtDt}{\,\, => Solution}
\]

\[ \footnote{University of Ibadan, ODE Integration Bee Finals, Problem 8} \lim_{n \to \infty} \frac{1}{n^2} \sum_{i=1}^{n} i \sin^3 \left( \frac{i \pi}{4n} \right)   \href{https://youtu.be/zl0HBW-bkFE?si=A0u41RcZ1E0VJ96-}{\,\, => Solution}
\]

\[ \footnote{University of Ibadan, ODE Integration Bee Finals, Problem 7} \int_0^{\infty} cos((\frac{x}{\pi}-\frac{e}{x})^2) dx   \href{https://youtu.be/2V3vS9c7Q-U?si=OS36ltwL0RzSVUDO}{\,\, => Solution} 
\]

\[ \footnote{ This textbook problem is misleading } Given that for an ideal gas: PV = nRT, prove that: \frac{\partial P}{\partial V}. \frac{\partial V}{\partial T}. \frac{\partial T}{\partial P} = -1   \href{https://youtu.be/EvXawwn4MK8?si=Ogmgp1JaTzkf7gZu}{\,\, => Solution}
\]

\[ \footnote{ The absolute perfectness} \int_{-\infty}^{\infty} erfc( (\frac{x}{e}-\frac{\zeta(3) \pi \gamma}{x})^2) dx = \frac{2e}{\sqrt{\pi}} \Gamma(\frac{3}{4})   \href{https://youtu.be/cGdQ5jOZ1JQ?si=drpfCMeF3OIjKolZ}{\,\, => Solution}
\]

\[ \footnote{ Summation Notation isn't just a tool, it's an emotion} \frac{1}{1^2} - \frac{2}{3^2} + \frac{3}{5^2} - \frac{4}{7^2} + ....   \href{https://youtu.be/Gw8vt5rMNhc?si=WGtoWwuVSA2Uu-NT}{\,\, => Solution}
\]

\[ \footnote{ Isn't this beautiful?} 1 + \frac{\cos(x)}{1!} + \frac{\cos(2x)}{2!} + \frac{\cos(3x)}{3!} +.........
= e^{\cos x} \cos(\sin x)   \href{https://youtu.be/NFnj7mS-0L0?si=ICRSsD6W_3H59lzg}{\,\, => Solution}  \]

\[ \footnote{Hehe} x^3 + y^3 + z^3 - 3xyz = (x+y+z)(x^2+y^2+z^2 - xy - yz - xz)}  \href{https://youtu.be/Fx01NS2DycQ?si=4LGIYIWJ92lLlNT0}{\,\, => Solution} 
\]

\[ \footnote{ x^3 + y^3 + z^3 - 3xyz} = (x+y+z) ( x + \omega ^2 y + \omega z) ( x + \omega y + \omega^2 z) \href{https://youtu.be/Fx01NS2DycQ?si=4LGIYIWJ92lLlNT0}{\,\, => Solution}  \]

\[ \footnote{ Explicit Content}Factorize:  x^2 + xy + y^2   \href{https://youtu.be/zC_JklxlU08?si=3oNerls_txS1j8mo}{\,\, => Solution}  \]

\[ \footnote{ Proof of the second order directional derivatives and more} 
D_vf = \cos\theta \frac{\partial f}{\partial x} +  \sin \theta \frac{\partial f}{\partial y}  \href{https://youtu.be/Id0iYXsX4Bc?si=soTqtxJk6GVth2gO}{\,\, => Solution} 
\]

\[ D^2_vf = \cos^2 \theta \frac{\partial
^2 f}{\partial x^2} + 2 \cos \theta \sin \theta \frac{\partial^2 f }{\partial x \partial y } + \sin^2 \theta \frac{\partial^2 f}{\partial y^2} \href{https://youtu.be/Id0iYXsX4Bc?si=soTqtxJk6GVth2gO}{\,\, => Solution} 

\]

\[ D_uD_vf = ( \cos\phi \partial_x +  \sin \phi \partial_y )  ( \cos\theta \partial_x +  \sin \theta \partial_y ) f   \href{https://youtu.be/Id0iYXsX4Bc?si=soTqtxJk6GVth2gO}{\,\, => Solution} \]

\[ D_wD_uD_vf = ( \cos\alpha \partial_x +  \sin \alpha \partial_y ) ( \cos\phi \partial_x +  \sin \phi \partial_y )  ( \cos\theta \partial_x +  \sin \theta \partial_y ) f   \href{https://youtu.be/Id0iYXsX4Bc?si=soTqtxJk6GVth2gO}{\,\, => Solution} \]

\[ \footnote{ Overkilling this proof using Fermat's last  theorem} \sqrt[n]{2} : n \ge 2 is irrational.  \href{video link here}{\,\, => Solution} 
\]

\[ \footnote{ A really elegant proof to prove at least one of e+pi or e.pi is irrational}  e + \pi , e \pi \href{https://youtu.be/k24FiBZW1Nw?si=ZaiDGw77AVHj-842}{\,\, => Solution} \]

\[ \footnote{ This is true in some world} (x+y)^p = x^p + y^p \href{https://youtu.be/scyft9YenzE?si=gRJnUWwTzeHMYMA_}{\,\, => Solution}  \]

\[ \footnote{ It's a prank} \int_0^{\frac{\pi}{4}} \sin(2x) \prod_{n=0}^{\infty} \left( e ^ { (-1)^n  (\tan(x))^{2n}} \right) dx   \href{https://youtu.be/cd0tY1rOIyM?si=APL15dkYRczg9bG2}{\,\, => Solution}  \]

\[ \footnote{ Austria Integration Bee 2024 Quarter Finals} \int \sum_{k=0}^{2024} \sin\left( x + \frac{2k.\pi}{2024} \right) dx \href{https://youtu.be/YObH9P5cszw?si=YsYlY89TgeKtz9BQ}{\,\, => Solution} 
\]

\[ \footnote{ Wohoo, Computing determinant with Induction, this is so fun} 
\begin{vmatrix}
a_1^2 + k & a_1 a_2 & a_1 a_3 & a_1 a_4 & \cdots & a_1 a_n \\
a_2 a_1 & a_2^2 + k & a_2 a_3 & a_2 a_4 & \cdots & a_2 a_n \\
a_3 a_1 & a_3 a_2 & a_3^2 + k & a_3 a_4 & \cdots & a_3 a_n \\
\vdots & \vdots & \vdots & \vdots & \ddots & \vdots \\
a_n a_1 & a_n a_2 & a_n a_3 & a_n a_4 & \cdots & a_n^2 + k
\end{vmatrix}  \href{https://youtu.be/vDv1WMZmHkA?si=HEBHuTns8-N8BTTx}{\,\, => Solution} 


\]

\[ \footnote{ Hmm, isn't that interesting} 
y'' - y' = \ln(t) ; y(0) = 0 ; y'(0) = 0  \href{https://youtu.be/-fTc2ugs6pg?si=GeSl5cN64i8tl597}{\,\, => Solution} 
\]

\[ y(t) = -\gamma e^t - \ln(t) - t \ln(t) + t + e^t Ei(-t)  \href{https://youtu.be/CvDGF2HRYPI?si=HjD53LCzh5F04Zho}{\,\, => Solution}  \]

\[ \footnote{Solving the Basel Problem using Fourier series of x^2} For -\pi < x < \pi,\]
\[ x^2 = \frac{\pi^2}{3} + \sum_{k=1}^{\infty} \frac{4 (-1)^k}{k^2} \cos(kx)   \href{https://youtu.be/hPPE-r5tM1Q?si=M3LHlhd-TtlejZWa}{\,\, => Solution} \]

\[ \footnote{ Introducing fourier transform from the fourier series} f(x) = \sum_{k=0}^{\infty} a_n \cos(\frac{n2\pi x}{T}) + \sum_{k=0}^{\infty} b_n \sin(\frac{n2\pi x}{T})
\]
\[ F(\omega)= \int_{-\infty}^{\infty} 
f(x) e^{-i\omega x} dx
  \href{https://youtu.be/G6_nSjLUai4?si=4DT_XN2oMXr98RZn}{\,\, => Solution} \]

\[ \footnote{ Solving Heat Equation PDE: which one is easier? i) Variable Separation Method ii) 
Fourier Transform Method} \frac{\partial u}{\partial t} = \alpha ^2 \frac{\partial^2 u}{\partial x^2} \]
\[ -\infty < x < \infty , t \ge 0 \]
\[ u(x,0) = f(x) , u(0,t)=0, u(L,t)=0   \href{https://youtu.be/YKN74-dO1OY?si=k91OXIvqH7CgOd_l}{\,\, => Solution}  \]
 
\end{document}


